\documentclass[11pt]{article}
\usepackage{listings}
\newcommand{\numpy}{{\tt numpy}}    % tt font for numpy

\topmargin -.5in
\textheight 9in
\oddsidemargin -.25in
\evensidemargin -.25in
\textwidth 7in

\begin{document}

% ========== Edit your name here
\author{Francesco Penasa}
\title{Bioinformatic resources - intro}
\maketitle

\medskip

% ========== Begin answering questions here
\texttt{2020 03 03}

\begin{lstlisting}[language=R]
# assigning values to variables
a <- 1 
b <- 2
c <- -1

# solving the quadratic equation
(-b + sqrt(b^2 - 4*a*c))/(2*a)
(-b - sqrt(b^2 - 4*a*c))/(2*a)

# HELP
help(function_name)

# create a sequence of numbers from 1 to 10
seq(1, 10) 

# sum all number in the function
sum(2, 3)

# log
log2(16)

# square root
sqrt(4)

# exponential
2^4

# e (e^2)
exp(2)
\end{lstlisting}


\begin{lstlisting}[language=R]
# loading package dslabs and the murders dataset
library(datasets)
data(cars)

# find the class of the object
class(cars)

# observe the structure
str(cars)

# show the head or tail
head(murders)
tail(murders)

# obtain the column `speed'
cars$speed

# obtain name of the columns
names(cars)

# obtain the values once 
vaules(cars[[``speed'']])

# length
length(cars$dist)

# obtain the column `speed' with brackets
cars[[``speed'']]

# check if equals
identical(a, b)

# return the occurrencies of unique elements (works with things on which levels works)
table(c(``a'', ``a'', ``b''))
\end{lstlisting}


\begin{lstlisting}[language=R]
# create a vector with concat function
vec1 <- c(1, 2, 3)
vec2 <- c(`a', `b', `c')

# 3d vector
vec_3d <- c(a=1, b=2, c=3)

# access second elem of array
codes[2]

# access elem 1 and 3
codes[c(1,3)]

# access elem from 1 to two (included)
codes[1:2]

# cast
as.character()
as.numeric()
\end{lstlisting}


\begin{lstlisting}[language=R]
# sort a vector
sort()

# produce a vector with the order of the elements
order()

# inverse of sort
rank()

# quantitative functions
max()
min()

# return the index
which.max()
which.min()
\end{lstlisting}



\begin{lstlisting}[language=R]
# create a logical vector
logical_vector <- grep(`a | e | u', vec)
\end{lstlisting}



\end{document}
\grid
\grid
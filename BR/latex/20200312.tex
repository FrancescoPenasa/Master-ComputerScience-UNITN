\documentclass[11pt]{article}
\usepackage{listings}
\newcommand{\numpy}{{\tt numpy}}    % tt font for numpy

\topmargin -.5in
\textheight 9in
\oddsidemargin -.25in
\evensidemargin -.25in
\textwidth 7in

\begin{document}

% ========== Edit your name here
\author{Francesco Penasa}
\title{Bioinformatics Resources - Introduction to Biological Databases}
\maketitle

\medskip

% ==========
\texttt{2020 03 12}
Amino acid sequences (DNA/RNA), protein sequences and more complex data stored efficiently in databases to be accessible by who needs them to analyze data and generate new knowledge.

\begin{enumerate}
	\item Primary databases: sequences of nucleotides and aminoacids
	\item Derived and specialized databases (protein domains, structures, genes...)
\end{enumerate}
Information gathered through literature, lab analyses and bioinformatics analyses.
Each database is characterized by a central biological element.
In primary data banks \textbf{each element is uniquely identified} by an accession number.
% paragraph biological_data (end)

\begin{enumerate}
	\item Sequences of nucleotides are represented by \textbf{4 characters} strings.
	\item Sequences of amino acids are represented by \textbf{20 characters} strings.
\end{enumerate}
$ACGT = DNA$ $ACGU = RNA$ amino acids are represented by a triplet of DNA alphabet, mapped into a new character.

\paragraph{Primary datatbases} % (fold)
\label{par:primary_datatbases}
data format differs
\begin{enumerate}
	\item GenBank: just store and arachive only nucleotide sequences. Every GenBank record is identified uniquely by the \textbf{ACCESSION and VERSION} codes.
	\item EMBL datalibrary
	\item DDBJ
\end{enumerate}
% paragraph primary_datatbases (end)

\paragraph{Derived data banks} % (fold)
\label{par:derived_data_banks}
\begin{enumerate}
	\item RefSeq: curated and not redundant collection of DNA, RNA and protein sequences.
\end{enumerate}
% paragraph derived_data_banks (end)

\textbf{All NCBI databases are accessible throught a nuque search engine called EWntrez} \texttt{https://www.ncbi.nlm.nih.gov/search/}

\paragraph{Genome browser} % (fold)
\label{par:genome_browser}
Allow us to browese data at various detail levels.

\subparagraph{Reading the gene structure} % (fold)
\label{subp:reading_the_gene_structure}
\begin{enumerate}
	\item Horizontal line with arrows : \textbf{INTRON}
	\item Dark block: \textbf{EXON}	
	\item Thick and arrowless lines: \textbf{UTR}
\end{enumerate}
% subparagraph reading_the_gene_structure (end)
% paragraph genome_browser (end)

\section{Exercises} % (fold)
\label{sec:exercises}
\begin{enumerate}
	\item Display the RARA gene with the browser. How many alternative transcripts does it have? \textbf{2}
	\item Considering the first transcript, how many introns and exons? \textbf{8 ; 9 }
	\item Now add to the displayed tracks the GC-percent track (it shows the percentage of GC bases along the sequence). Drag it and move it just under the sequence. \textbf{Mapping and Sequencing GC percent dense}
	\item Now hide the alignment track which is shown by default. \textbf{boh}
	\item Zoom out on the 5’UTR region of the transcript and check if there are any known SNPs in that region.
\end{enumerate}
\begin{enumerate}
	\item We want to retrieve the 3’UTR sequence for our gene, RARA
	\item \texttt{http://genome.ucsc.edu/cgi-bin/hgTables}
	\item identifiers paste list -> RARA
	\item output format -> sequence, genomic
	\item change some things and submit
\end{enumerate}
 
\begin{enumerate}
	\item Display the KRAS gene with the browser. How many alternative transcripts does it have ? Considering the first transcript, how many introns and exons? 4; 5; 6
	\item Now add the 1000 Genomes – EUR common variants track to the display. Are there 5'UTR variants in KRAS? 
	\item Now zoom to the second exon of the first isoform of KRAS. How many missense variants are there ?
\end{enumerate}

\begin{enumerate}
	\item Get the Ensembl ID and MGI symbol for all mouse genes on chromosome 19. How many genes are there ?  
	\item Now limit this to protein coding genes. How many genes are there ? 
	\item Now save the results to a comma-separated values fil
\end{enumerate}
% section exercises (end)

\end{document}
\grid
\grid
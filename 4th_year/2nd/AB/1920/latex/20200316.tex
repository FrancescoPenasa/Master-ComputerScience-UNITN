\documentclass[11pt]{article}
\usepackage{listings}
\usepackage{color}

\usepackage{graphicx}
\graphicspath{ {./images/} }
% R style
\lstset{frame=tb,
language=R,
keywordstyle=\color{blue},
alsoletter={.}
}

\newcommand{\numpy}{{\tt numpy}}    % tt font for numpy

\topmargin -.5in
\textheight 9in
\oddsidemargin -.25in
\evensidemargin -.25in
\textwidth 7in

\begin{document}

% ========== Edit your name here
\author{Francesco Penasa}
\title{Algorithms for Bioinformatics - lect 6-7}
\maketitle

\medskip

% ========== Begin answering questions here
\texttt{2020 03 16}
\section{Previous lectures} % (fold)
\label{sec:previous_lectures}
\begin{enumerate}
	\item Global sequence alignment
	\item Local sequence alignment	
	\item substitution matrices (PAM and BLOSUM)
\end{enumerate}
% section previous_lectures (end)

\section{Issue about the complexity} % (fold)
\label{sec:issue_about_the_complexity}
\paragraph{Needleman-Wunsch} % (fold)
\label{par:needleman_wunsch}
time
\[
	O(mn)
\]
space 
\[
	O(n)
\]
$m$ and $n$ are the length of the two sequences

\paragraph{Smith-Waterman} % (fold)
\label{par:smith_waterman}
time 
\[
	O(mn^2) 
\]
$n^2$ because gaps of variable length are considered during the computation of each element of the matrix. It is possible to reduce Local alignment in $O(mn)$

\paragraph{} % (fold)
\label{par:0}
Is the complexity low enough for pratical uses? No. Use Heuristics methods, sacrifice exact solution in order to gain in terms of time complexity.
% paragraph  (end)

% section issue_about_the_complexity (end)

\section{FASTA} % (fold)
\label{sec:fasta}
FASTA is an heuristic method for global/local sequence alignment.
\begin{enumerate}
	\item Look-up table: between the sequences
	\item Best initial regions found: through the ktup matrix
	\item Connect the best initial regions heuristically 
	\item Computing an exact alignment limiting the search to a promising area
\end{enumerate}
% section fasta (end)
\texttt{https://www.ebi.ac.uk/Tools/sss/fasta/}
\section{BLAST} % (fold)
\label{sec:blast}  
BELLO
% section blast_and_its_variations (end)

\end{document}
\grid
\grid
\documentclass[11pt]{article}
\usepackage{listings}
\newcommand{\numpy}{{\tt numpy}}    % tt font for numpy

\topmargin -.5in
\textheight 9in
\oddsidemargin -.25in
\evensidemargin -.25in
\textwidth 7in

\begin{document}

% ========== Edit your name here
\author{Francesco Penasa}
\title{LCSE - homework for lect 3}
\maketitle

\medskip

% ========== Begin answering questions here

\texttt{Report of a Workshop on The Scope and Nature of Computational Thinking}
\begin{enumerate}
	\item Nell'informatica e nella matematica viene usata una struttura linguistica per descrivere precisamente senza ambiguita', come nei metodi formali e come si cerca di fare in ingegneria del software con la descrizione dei requisiti. (confermato anche poi di aunto detto nel sottocapitolo successivo nella parte di musica e computational thinking e inn law)
	\item Insegnare i circuiti partendo dal da una soluzione fatta bene preparata per risolvere questi problemi come un esperto farebbe, insegnando di fatto il ragionamente fatto da un esperto.
	\item Gli esempi in diversi campi (medico archeologico etc) ci fanno pensare che il computational thinking sia necessario soprattutto dove sono presenti grandissime quantita di dati che un essere umano non puo' analizzare singolarmente. Per appunto astrarre dei concetti troppo complicati e renderli analizzabili facilmente.
	\item Una soluzione all'ambiguita' linguistica.
	\item Il modo di trovare le informazioni importanti grazie al pensiero computazionale tra la grandissima quantita' di informazioni disponibili.
\end{enumerate}


\section{Secondo capitolo} % (fold)
\label{sec:secondo_capitolo}
\begin{enumerate}
	\item studio dei meccanismi dell'intelligenza che possono portare ad applicazioni pratiche aumentando l'intelligenza umana.
\end{enumerate}


\section{Terzo capitolo} % (fold)
\label{sec:terzo_capitolo}
\begin{enumerate}
	\item lato matematico: la struttura linguistica a pagina 33
	\item lato ingegneristico: il bisogno di limiti
\end{enumerate}
% section terzo_capitolo (end)

\section{Quarto capitolo} % (fold)
\label{sec:quarto_capitolo}
\begin{enumerate}
	\item 
\end{enumerate}
% section quarto_capitolo (end)

\end{document}
\grid
\grid


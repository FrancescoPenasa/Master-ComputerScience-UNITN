\documentclass[11pt]{article}
\usepackage{listings}
\newcommand{\numpy}{{\tt numpy}}    % tt font for numpy

\topmargin -.5in
\textheight 9in
\oddsidemargin -.25in
\evensidemargin -.25in
\textwidth 7in

\begin{document}

% ========== Edit your name here
\author{Francesco Penasa}
\title{Laboratory of Computer Science Education}
\maketitle

\medskip

% ========== Begin answering questions here
\texttt{http://cricca.disi.unitn.it/montresor/teaching/lcse/letture/}\\

\section{Qual e' la natura dell'informatica} % (fold)
\label{sec:qual}
Per definire cosa sia l'informatica:
Attenzione rivolta a come l'informatica viene insegnata in termini di contenuti e di approcci meotodologici, si presume che i modelli considerati riflettano sia le tematiche di ricerca, sia la pratica professionale.

Vengono inoltre considerati il dibattito epistemologico e filosofico.

Mi trovo daccordo con il rationalist paradigm 
Sono in disaccordo con il technocratic paradigm, penso che le argomentazioni che porta non siano corrette, ad esempio in epistemology impratical to specify formally or to prove deductively the correctness of a complete program. Esempio corso di formal methods  
% section qual\ (end)



\end{document}
\grid
\grid
\documentclass[11pt]{article}
\usepackage{listings}
\newcommand{\numpy}{{\tt numpy}}    % tt font for numpy

\topmargin -.5in
\textheight 9in
\oddsidemargin -.25in
\evensidemargin -.25in
\textwidth 7in

\begin{document}

% ========== Edit your name here
\author{Francesco Penasa}
\title{Laboratory of Computer Science Education}
\maketitle

\medskip

% ========== Begin answering questions here
\texttt{2020 03 24}

\section{Letture} % (fold)
\label{sec:letture}
\paragraph{Gears of your childhood} % (fold)
\label{par:gears_of_your_childhood}
\begin{enumerate}
	\item giochi da tavolo: sistema formale di regole (sistema unplugged per il pensiero computazionale)
	\item lego
	
\end{enumerate}
% paragraph gears_of_your_childhood (end)


\texttt{https://learning.media.mit.edu/content/publications/EA.Piaget \_ Papert.pdf}
\paragraph{Piaget’s Constructivism, Papert’s Constructionism: What’s the difference?} % (fold)
\label{par:Constructivism}

% paragraph  (end)

\texttt{https://drive.google.com/file/d/1JVpQ5aRBmnXu-w\_tjUdQweWfPxQEMaqc/view}
\paragraph{Constructionism} % (fold)
\label{par:Constructionism}
Constructionism: learning with discovery learning, discover principles or ideras by him or herself.

% section letture (end)

\section{Spunti di ragionamento / discussione} % (fold)
\label{sec:spunti_di_ragionamento_discussione}
\begin{enumerate}
	\item Quali sono i “gears of your childhood”? Ricordate episodi di apprendimento significativo motivati dal vostro particolare interesse?
	\item Per chi è abituato a parlare di algoritmi, teoremi, dimostrazioni, molti dei contenuti di questi articoli possono sembrare chiacchiere ad alto livello. Ci aspettiamo un insieme di istruzioni più precise, come “Come porto questi concetti in classe?”. Lo chiedo a voi: \begin{enumerate}
		\item Riuscite ad identificare metodologie e suggerimenti presi da questi testi che possano essere immediatamente applicati?
		\item Riuscite ad identificare metodologie e suggerimenti di più ampio respiro?
		\item Ci raccontate qualche esempio significativo di insegnanti che, anche inconsciamente, seguivano questa scuola di pensiero?
		\item Qual è la vostra opinione sul pensiero di Papert?
	\end{enumerate}
	\item Vorrei che rifletteste anche sul particolare stile di Papert per quanto riguarda i risultati dei suoi interventi. Tanti aneddoti, nessun dato.  Ne parliamo in questa lezione e poi ritorneremo su questi aspetti quando affronteremo la ricerca sulla didattica dell’informatica.
	\item Infine, vorrei che meta-rifletteste sulla natura delle mie lezioni alla luce di quanto avete letto. Sto usando, o almeno tentando di usare, un approccio costruttivista alla insegnamento dell’insegnamento?
\end{enumerate}
% section spunti_di_ragionamento_discussione (end)
\end{document}
\grid
\grid
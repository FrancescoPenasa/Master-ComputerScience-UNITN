\documentclass[11pt]{article}
\usepackage{listings}
\newcommand{\numpy}{{\tt numpy}}    % tt font for numpy

\topmargin -.5in
\textheight 9in
\oddsidemargin -.25in
\evensidemargin -.25in
\textwidth 7in

\begin{document}

% ========== Edit your name here
\author{Francesco Penasa}
\title{Agent-Oriented Software Engineering - 01 Introduction to the Course}
\maketitle

\medskip

% ========== Begin answering questions here
\texttt{2020/02/17}
\paragraph{Programming progression} % (fold)
\label{par:programming_progression}
From machine code iterationing of abstractions to agents. 
\begin{enumerate}
	\item machine code
	\item assembly language
	\item procedures \& functions
	\item objects
	\item services
	\item agents
\end{enumerate}
% paragraph programming_progression (end)

\paragraph{Agent, a definition:} % (fold)
\label{par:agent_a_definition_}
An agent is a computer system that is capable of \textbf{independent} actions on behalf of its user or owner
% paragraph agent_a_definition_ (end)

\paragraph{Multi-Agent System, a definition} % (fold)
\label{par:multi_agent_system_a_definition}
\begin{enumerate}
	\item consists in different agents that interact with one-another
	\item agents act on behalf of users with different goals and motivations
	\item agents will cooperate, coordinate and negotiate with each other to success
\end{enumerate}
% paragraph multi_agent_system_a_definition (end)

\paragraph{A good Agent-Oriented Software Engineering Methodology} % (fold)
\label{par:a_good_agent_oriented_software_engineering_methodology}
\begin{enumerate}
	\item modeling languages
	\item analysis techniques
	\item design techniques
	\item supporting tools
\end{enumerate}
% paragraph a_good_agent_oriented_software_engineering_methodology (end)

\end{document}
\grid
\grid
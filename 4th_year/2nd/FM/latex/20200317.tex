 \documentclass[11pt]{article}
\usepackage{listings}
\newcommand{\numpy}{{\tt numpy}}    % tt font for numpy

\topmargin -.5in
\textheight 9in
\oddsidemargin -.25in
\evensidemargin -.25in
\textwidth 7in

\begin{document}

% ========== Edit your name here
\author{Francesco Penasa}
\title{Formal methods - Temporal logic}
\maketitle

\medskip

% ========== Begin answering questions here
\texttt{2020 03 17}
\section{CTL model checking} % (fold)
\label{sec:ctl_model_checking}

\subsection{general ideas} % (fold)
\label{sub:general_ideas}

CTL Model Checking is a formal verification technique where:
\begin{enumerate}
	\item the system is represented as a \textbf{Finite State Machine} $M$
	
	\item the property is as CTL formula $\phi$ \[
		AG(p \rightarrow AF q)
	\] in all possible initial states, in all possible paths starting from the initial state (AG = in all reachable states), if p holds then for all path sooner or later we hold q.

	\item the model checking algorithm checks whether in \textbf{all initial states} of M \textbf{all the executions} of the model \textbf{satisfy} the formula ($M \models \phi$).
\end{enumerate}

In general it works in two macro steps:
\begin{enumerate}
	\item construct the set of states where the formula holds:\[
		[\phi] := {s \in S : M,s \models \phi}
	\] ($[\phi]$ is called the \textbf{denotation} of $\phi$)

	\item then compare with the set of \textbf{initial states}:\[
		I \subseteq [\phi] ?
	\]
\end{enumerate}
In order to compute $[\phi]$ we proceed bottom up, this way. 
\begin{enumerate}
	\item $[q]$
	\item $[AFq]$
	\item $[p]$
	\item $[p \rightarrow AFq]$
	\item $[AG (p \rightarrow AFq)]$
\end{enumerate}
To compute each $[\phi _i]$
\begin{enumerate}
	\item labeling function: in which state a given proposition holds.
	\item use set operations
	\item compute \textbf{pre-images}
	\item applying \textbf{tableaux rules}, until a fixpoint is reached: recursive definition of the propostitions
\end{enumerate}

% subsection general_ideas (end)
\subsection{A simple example} % (fold)
\label{sub:a_simple_example}
\begin{enumerate}
	\item $q$ We find the states where q holds : state 2
	\item $AF q$ recall the AF tableau rule $AF q \leftrightarrow (q \lor AXAF q)$
\end{enumerate}
% subsection a_simple_example (end)
\subsection{Some theoretical issues} %  (fold)
\label{sub:some_theoretical_issues}
stuff


% subsection some_theoretical_issues (end)
\subsection{Algorithms} % (fold)
\label{sub:algorithms}
Assume using only $ \lnot \land EX EU EG$
\begin{enumerate}
	\item for every $\phi _i \in Sub(\phi)$, find $[\phi _i]$
	\item Check if I $\subseteq [\phi]$
\end{enumerate}
\begin{enumerate}
	\item propositional atoms: apply labeling function
	\item boolean operator: apply standard set operations
	\item temporal operator: apply recursively the tableaux rules, until a \textbf{fixpoint} is reached.
\end{enumerate}

\begin{lstlisting}[mathescape=true]
state_set Check(CTL_formula $\beta$){
	case $\beta$ of
	true: return S;
	false: return {};
	$\lnot \beta_1$: return S \ Check($\beta_1$);
	$\beta_1 \land \beta_2$: return Check($\beta_1$) $\cap$ Check($\beta_2$);
	$EX\beta_1$: return PreImage(Check($\beta_1$));
	$EG\beta_1$: return Check_EG(Chekck($\beta_1$));
	$E(\beta_1 U \beta_2)$: return Check_EU(Check($\beta_1$), Check($\beta_2$));
}
\end{lstlisting}

\paragraph{PreImage} % (fold)
\label{par:preimage}
Add to the S if it is a successor and it holds 
% paragraph preimage (end)
\paragraph{Check\_EG} % (fold)
\label{par:check_eg}
Intersect to the S if it is a successor and it holds 
% paragraph check_eg (end)
\paragraph{Check\_EU} % (fold)
\label{par:check_eu}
Add to the S if it Intersect is a successor and it holds
% paragraph check_eu (end)
\paragraph{Check\_EF} % (fold)
\label{par:check_ef}

$$
	X' := X \cup PreImage(X)
$$

% paragraph check_ef (end)



% subsection algorithms (end)
\subsection{Some examples} % (fold)
\label{sub:some_examples}
\subsubsection{Mutual exclusion} % (fold)
\label{ssub:mutual_exclusion}
INSERT figure
\[
	M \models AGAF C_1 ?
\]
To use the algorithm we rewrite it in terms of EG
\[
	\Rightarrow M \models \lnot EFEG \lnot C_1 ?
\]
Compute the denotations of 
\[
	[\lnot C_1]
\]
easy. Now this
\[
	[EG \lnot C_1]
\]
We take our current set of states and \textbf{intersect} with its preimage. Drop all states that does not hold this $EX  S$ next step $EX S$, repeat until there are no changes. Fixpoint reached!
\[
	[EFEG \lnot C_1]
\]
We do the \textbf{union} of our current approximation with the preimage. $EX S$, repeat until we reach a fixed point.
\[
	[\lnot EFEG \lnot C_1]
\]
Easy af. we can prove that the property is not verified.


\subsubsection{liveness} % (fold)
\label{ssub:liveness}
\[
	M \models AG(T_1 \rightarrow AF C_1)? \Rightarrow M \models \lnot EF (T_1 \land EG \lnot C_1)?
\]
% subsubsection liveness (end)

\subsubsection{Complexity of CTL model checking} % (fold)
\label{ssub:complexity_of_ctl_model_checking}
$O(|\phi|)$ steps $O(|M|)$ states to explore. $O(|M| * |\phi|)$
Typically the number of states is huge.

% subsubsection complexity_of_ctl_model_checking (end)


% subsection some_examples (end)
\subsection{Subcase: invariants} % (fold)
\label{sub:subcase_invariants}
Invariant properties have the form $AG p$ (e.g., $AG \lnot bad$). To check that a bad state is not reachable ($AG \lnot bad == \lnot EF bad$).
In this very particular case we can reason that it is possible to stop when we intersect with the initial step OR when we reach a fixed point. But we can do better.
\subsubsection{Symbolic forward model checking of invariants} % (fold)
\label{ssub:symbolic_forward_model_checking_of_invariants}
\textbf{Forward checking}
\begin{enumerate}
	\item compute [bad]
	\item compute the set of initial states $I$
	\item compute the set of reachable states from $I$ until it intersect [bad] or a fixed point is reached.
\end{enumerate}
Basic step is the \textbf{Forward Image}\[
	Image(Y) := {s' | s \in Y and R(s,s') holds}
\]
Simplest form: compute the set of reachable states.
% subsubsection symbolic_forward_model_checking_of_invariants (end)
% subsection subcase_invariants (end)

\subsection{Exercises} % (fold)
\label{sub:exercises}
\paragraph{CTL model checking} % (fold)
\label{par:ctl_model_checking}
$\phi = AG ((p \land q) \rightarrow EG q)$
TODO ADD figure\\
Rewrite $\phi$ into an equivalent formula $\phi'$ expressed in terms of EX, EG, EU/ EF only.\\
Compute bottom-up the denotations
\[
	p \land q = s_1
\]\[
	q = s_1 \cup s_0
\]
next step
\[
	EG q = s_1 \cup s_0 = {s_0, s_1} 
\]
next step
\[
	\lnot EG q = {s_2} 
\]
next steps
\[
	(p \land q) \land (\lnot EG q) = \emptyset
\]
next step 
\[
	EF((p \land q) \land (\lnot EG q)) = \emptyset
\]
next step 
\[
	\lnot EF((p \land q) \land (\lnot EG q)) = S = s_1, s_2, s_3
\]
finished.\\
As consequence of the previous point, say wheter $M \models \phi$ \textbf{Yes}

% paragraph ctl_model_checking (end)

\paragraph{CTL model checking 2} % (fold)
\label{par:ctl_model_checking_2}
\[
	AG(AF p \rightarrow q)
\]
Rewrite
\[
	pirpippir
\]
Compute time

\[
	\lnot p = s_2, s_1
\]
\[
	EG \lnot p = s_1, s_2 \cup s_0, s_1, s_2  = s_1, s_2
\]
\[
	\lnot EG \lnot p = s_0
\]
\[
	\lnot q = s_1
\]
\[
	EG \lnot q = s_1 \cup s_0, s_1, s_2 = s_1		
\]
\[
	(\lnot EG \lnot p) \land (EG \lnot q) = s_1 \cup  s_0, s_2 = \emptyset
\]
\[
	EF((\lnot EG \lnot p) \land (EG \lnot q)) = \emptyset
\]
\[
	\lnot EF((\lnot EG \lnot p) \land (EG \lnot q)) = s_0, s_1, s_2		
\]

% paragraph ctl_model_checking_2 (end)
% subsection exercises (end)
% section ctl_model_checking (end)

\section{Homework} % (fold)
\label{sec:homework}
Apply the same process to all the CTL examples of Chapter 3 
EX = pre image;  EG = intersection, EF equal Union to what we have; E(a U b) = b U (a inter image(a))
% section homework_ (end)
\end{document}
\grid
\grid


% slide 04 27/71 temporal operator: APPY
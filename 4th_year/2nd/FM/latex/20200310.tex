\documentclass[11pt]{article}
\usepackage{listings}
\newcommand{\numpy}{{\tt numpy}}    % tt font for numpy

\topmargin -.5in
\textheight 9in
\oddsidemargin -.25in
\evensidemargin -.25in
\textwidth 7in

\begin{document}

% ========== Edit your name here
\author{Francesco Penasa}
\title{Formal methods - Temporal logic}
\maketitle

\medskip

% === Starting point === %
\texttt{2020 03 10}
\section{Summary} % (fold)
\label{sec:summary}
We have seen the behaviour of Kripke structure as \textbf{infinite set of computation paths} and \textbf{infinite computation tree}.
\paragraph{LTL} % (fold)
\label{par:ltl}
extension of boolean logic with X(next) U(until) G(globally) F(finally) R(releases, give the permission to be false).
LTL properties are evaluated over single paths.
$G\phi$ is stronger than anyone.
% paragraph ltl (end)
% section summary (end)

\subsection{LTL tableaux rules} % (fold)
\label{sub:ltl_tableaux_rules}
Let $\phi _1$ and $\phi_2$ be LTL formulae:
\[
	G\phi _1 \Leftrightarrow (\phi _1 \lor XF\phi _1)
\]\[
	F\phi _1 \Leftrightarrow (\phi _1 \land XG\phi _1)
\]\[
	\phi _1 U \phi_2 \Leftrightarrow (\phi_2 \lor (\phi _1 \land X(\phi _1 U \phi_2)))
\]\[
	\phi _1 R \phi_2 \Leftrightarrow (\phi_2 \land (\phi _1 \lor X(\phi _1 R \phi_2)))
\]
If appllied recursively, rewrite an LTL formula in terms of atomic and $X$-formulas:
\[
	(pUq)\land(G\lnot p) \Rightarrow (q \lor (p \land X(p U q))) \land (\lnot p \land XG \lnot p)
\]
Extremely important.
\subsection{Some property} % (fold)
\label{sub:some_property}
\begin{enumerate}
	\item Fairness 
\end{enumerate}

\[
	G(T \rightarrow)
\]
% subsection some_property (end)
% subsection ltl_tableaux_rules (end)
\section{Computation Tree Logic CTL} % (fold)
\label{sec:computation_tree_logic_ctl}
\subsection{Syntax} % (fold)
\label{sub:syntax}
An atomic proposition is a CTL formula, as all his combinations plus the combination with the 8 temporal operators in CTL, AX, AU, AG, AF, EX, EU, EG, EF.\\
$A = $ Necessarily, it applies to all path starting from the current point
$E = $ Possibly, it applies to at least one path starting from the current point.
\textbf{Remember} that in this case we thinking in tree, when we say every possible successor we consider all the successors of a certain node.
\begin{enumerate}
	\item AX necessarily next
	\item EX possibly next
	\item AF necessarily in the future (inevitably)
	\item EF possibly in the future (possibly)
	\item AG always true  in all paths
	\item $EG\phi$ there is at least one path where $\phi$ is always true.
	\item $A(\phi U \psi)$ no matter what sooner or later in every path $\phi$ will hold.
	\item $E(\phi U \psi)$ 
\end{enumerate}
The figure is clearly self explaining. 03 60/108 
While in LTL we use a state in a given path for the definition, here we usa a given state $s_i$ and a given model $M$ (situation).
To sum up CTL properties are evaluated over trees. Universal modalities(AF,AG,AX,AU), Existential modalities(EF,EG,EX,EU).
\textbf{It is based on the pair $M,s_i$ called also a "situation".}\\
The CTL model checking problem $M \models  \phi$
$M,s \models \phi$ \textbf{for every initial state} $s \in l$ of the Kripke structure.
% subsection syntax (end)
\subsection{CTL tableaux rules} % (fold)
\label{sub:ctl_tableaux_rules}
Let $\phi_1$ and $\phi_2$ be CTL formulae:
\begin{enumerate}
	\item $AF\phi_1 \Leftrightarrow (\phi_1 \lor AXAF \phi_1)$
	\item $AG\phi_1 \Leftrightarrow (\phi_1 \land AXAG \phi_1)$ 
	\item $A(\phi_1 U \phi_2) \Leftrightarrow (\phi_2 \lor (\phi_1 \land AXA (\phi_1 U \phi_2))$
	\item $EF\phi_1 \Leftrightarrow (\phi_1 \lor EXEF \phi_1)$
	\item $EG\phi_1 \Leftrightarrow (\phi_1 \land EXEG \phi_1)$ 
	\item $E(\phi_1 U \phi_2) \Leftrightarrow (\phi_2 \lor (\phi_1 \land EXE (\phi_1 U \phi_2))$
\end{enumerate}
Recursive definitions of AF, AG, AU, EF, EG, EU.
If appllied recursively, rewrite a CTL formula in terms of atomic, AX- and EX- formulas:
\[
	A(pUq) \land (EG\lnot p) \Rightarrow (q \lor (p \land AXA(pUq))) \land (\lnot p \land EXEG \lnot p)
\]
% subsection ctl_tableaux_rules (end)
% section computation_tree_logic_ctl (end)

\subsection{Examples} % (fold)
\label{sub:examples}
\paragraph{} % (fold)
\label{par:}
TODO: add the figure \\
Mutual exclusion
\[
	M \models AG \lnot (C_1 \land C_2) ?
\] YES\\

Liveness
\[
	M \models AG (T_1 \rightarrow AF\ C_1) ?
\]
YES \\

Fairness
\[
	M \models AGAF\ C_1 ?
\]
NO \\

Fairness(2)
\[
	M \models AGAF\ turn=0 ?
\]
NO \\

Blocking
\[
	M \models AG (N_1 \rightarrow EF\ T_1) ?
\]
YES \\

Blocking(2)
\[
	M \models AG (N_1 \rightarrow AF\ T_1) ?
\]
NO \\

example 6
\[
	M \models EG\ N_1?
\]
YES \\

example 7
\[
	M \models AFEG\ N_1?
\]
YES \\

If we use $E$ we can observe that there is not a corresponding LTL formula.
% subsection examples (end)


\end{document}
\grid
\grid
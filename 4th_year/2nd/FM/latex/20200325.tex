\documentclass[11pt]{article}
\usepackage{listings}
\newcommand{\numpy}{{\tt numpy}}    % tt font for numpy

\topmargin -.5in
\textheight 9in
\oddsidemargin -.25in
\evensidemargin -.25in
\textwidth 7in

\begin{document}

% ========== Edit your name here
\author{Francesco Penasa}
\title{Formal methods - Temporal logic}
\maketitle

\medskip

% ========== Begin answering questions here
\texttt{2020 03 03}

We can see Kripke structure as a infinite set of computation paths and as an infinite computation tree.

\section{LTL} % (fold)
\label{sec:ltl}
When we reason on LTL we reason on a signle path.

\begin{enumerate}
	\item $X$ : next, $X\phi$ is true iff $\phi+1$ is true
	
	\item $G$ : globally $G\phi$ is true iff $\phi$ is true \textbf{from now} on \textbf{forever}
	
	\item $F$ : finally, $F\phi$ is true if sooner or later $\phi$ will be true, it could be also the current state.

	\item $U$ : until, $\phi U \psi$ is true if sooner or later $\psi$ is true (even now, and it must be true sooner or later) \textbf{AND} $\phi$ is true in all states until that.

	\item $R$ : releases $\phi R \psi$ is true iff for all states following this $\psi$ is true forever \textbf{OR} $\phi$ is true. $\phi$ authorize $\psi$ to not hold, phi 
\end{enumerate}

$\models$ means models
note that we are focusing on a particular state to model the future.

We can say that something holds in a path \textbf{if} it holds in all possible initial state.

for every path $\pi$ of the Kripke structure $M$
\[
	\pi \models \phi
\] 
N.B.
\[
	M \not\models \phi \not\Rightarrow M \models \lnot \phi
\]
we can see the example in the slides for this.

slides 44 \textbf{FONDAMENTAL!}

if we say something in an infinite path we break it in now and which property i have to satisfie the next step.
% section ltl (end)

$M \models T_1 R \lnot C_1$
either c1 is always false or in order for c1 to become true t1 has to become true.
t1 authorize c1


\section{CTL} % (fold)
\label{sec:ctl}
We work on the branching model of time
% section ctl (end)
\end{document}
\grid
\grid
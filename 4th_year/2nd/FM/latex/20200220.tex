\documentclass[11pt]{article}
\usepackage{listings}
\newcommand{\numpy}{{\tt numpy}}    % tt font for numpy

\topmargin -.5in
\textheight 9in
\oddsidemargin -.25in
\evensidemargin -.25in
\textwidth 7in

\begin{document}

% ========== Edit your name here
\author{Francesco Penasa}
\title{Formal Methods - 02\_Modeling\_Transistion\_Systems}
\maketitle

\medskip

% ================ %
lect 02 material 2020/02/20: 
\\
HANDOUTS : \texttt{http://disi.unitn.it/\~{}rseba/DIDATTICA/fm2020/02\_TRANSITION\_SYSTEMS\_HANDOUT.pdf}
\\
SLIDES : \texttt{http://disi.unitn.it/\~{}rseba/DIDATTICA/fm2020/02\_TRANSITION\_SYSTEMS\_SLIDES.pdf} 
\\

\section{Transistion Systems as Kripke Models} % (fold)
\label{sec:transistion_systems_as_kripke_models}
\paragraph{Kripke models are used to describe reactive systems:} % (fold)
\label{par:kripke_models_are_used_to_describe_reactive_systems_}
\begin{enumerate}
	\item nonterminating systems with \textbf{infinite} behaviours
	\item \textbf{dynamic evolution} of modeled systems;
	\item \textbf{state:} snapshot of the system (value of vars, program counter, etc.)
	\item \textbf{realistic representation}, the system can be simulated and validated before the implementation
\end{enumerate}
% paragraph kripke_models_are_used_to_describe_reactive_systems_ (end)


\paragraph{Kripke model: formal definition} % (fold)
\label{par:kripke_model_formal_definition}
\[
	<S,I,R,AP,L>	
\]
\begin{enumerate}
	\item $S =$ a \textbf{finite} set of states
	\item $I =$ initial states 
	\item $R =$ set of \textbf{directed} transisitions rules $R = S \dot S$
	\item $AP =$ \textbf{observable variables} (boolean variables)  
	\item $L =$ labeling function $L : S = \rightarrow 2^{AP}$; tells us what is the value of the boolean variables for each states 
	\item $R$ is Total, all states have an outgoing link, we don't consider sink states.
\end{enumerate}

\textbf{In Kripke structures the value of every variable is always assigned in each state.}
\textbf{Each state identifies univocally the values of the atomic propositions which hold there.}
% paragraph kripke_model_formal_definition (end)

\paragraph{TODO INSERT IMAGE GRAPH} % (fold)
\label{par:todo_insert_image_graph}
\[
	N,T,C == observable\_variables	
\]
\begin{enumerate}
	\item N = not in critical section and not trying to acquire the lock
	\item turn = who has the rigth to enter the critical section
	\item arc = step = when one process makes a move
\end{enumerate}
A \textbf{path} is an \textbf{infinite} sequence of states
% paragraph todo_insert_image_graph (end)

\paragraph{Path in a Kripke Model} % (fold)
\label{par:path_in_a_kripke_model}

A \textbf{path} in a Kripke model $M$ is an infinite sequence of states.  It must start from an initial state and use existing transition rules. 
\[
	\pi = s_0, s_1, s_2, \ldots \in S^{\omega}
\]
such that $s_0 \in I$ and all other $s_i \in R$ \\
\textbf{Reachable} if there is a path from the initial state to the target state.\\
% paragraph path_in_a_kripke_model (end)

\subsection{Composing Kripke Models} % (fold)
\label{sub:composing_kripke_models}

Complex Models are obtained by composition of smaller ones.
Both compositions are associative. 

\subsubsection{Asynchronous Composition} % (fold)
\label{ssub:asynchronous_composition}
At each time istant, one component is selected to perform a transition (for instance communication protocols).
The composition is obtained as all the possible interleaving possibilities between two models, \textbf{no contemporary moves}. The time is continous and nothing can happen in the same istant. \textbf{We must be really careful about the shared variables}. The main bounds are about the initial model and the respect to the \textbf{shared}observable variable.
\[
	M_1 || M_2 || M_3
\]
% subsubsection asynchronous_composition (end)
\subsubsection{Synchronous Composition} % (fold)
\label{ssub:synchronous_composition}
Components evolve in parallel, fixed time slots in which all the component perform a trasition (for instance a sequential hardware circuit).
The boundaries are the same as the asynchronous state with the only difference about the transition rules.
\[
	M_1 \times M_2 \times M_3
\]

% subsubsection synchronous_composition (end)
% subsection composing_kripke_models (end)


% section transistion_systems_as_kripke_models (end)
\section{Languages for Transition Systems} % (fold)
\label{sec:languages_for_transition_systems}

We only reason in therms of evolution of variables, we don't care or need the graph representation. We care about the possible moves.
A Kripke model is usually presented in a structured language.
\paragraph{Each component is presented by specifying} % (fold)
\label{par:each_component_is_presented_by_specifying}
\begin{enumerate}
	\item state variables ($AP$, $S$ and the map function $L$)
	\item initial values for state variables (initial state)
	\item instruction (transitions)
\end{enumerate}
% paragraph each_component_is_presented_by_specifying (end)
\subsection{SMV (NoSMV and NoXMV)} % (fold)
\label{sub:smv}\
No sequence, no implicit order (no program counter), we write the relations.
\begin{enumerate}
	\item booleans in many forms
	\item declarations of state variable; assignment for the initial state; assignment for the transition relation
	\item allows both synchronous and asynchronous composition (more sync)
\end{enumerate}
Big systems are represented in modular parts
% subsection svm (end)
\subsection{PROMELA} % (fold)
\label{sub:promela}
Sequential execution
\begin{enumerate}
	\item C-like
	\item set of processes that interacts with \textbf{shared variables} and \textbf{communication channels}
	\item allows both synchronous and asynchronous composition (more async)
\end{enumerate}

% subsection promela (end)
\subsection{SDL} % (fold)
\label{sub:sdl}
For infinite state machine and usage of continuos time.
\begin{enumerate}
	\item booleans in many forms
	\item allows TIME representations
	\item represents states, message I/O, conditions, clock operations, subroutines
	\item allows both synchronous and asynchronous composition (async)
\end{enumerate}
% subsection sdl (end)


% section languages_for_transition_systems (end)
\section{Properties of Transition Systems} % (fold)
\label{sec:properties_of_transition_systems}
\subsection{Safety Properties} % (fold)
\label{sub:safety_properties}
Bad events never happens
\begin{enumerate}
	\item Deadlock freedom
	\item Mutual Exclusion
\end{enumerate}
\textbf{Can be refuted by a \textit{finite} behaviour}
% subsection safety_properties (end)

\subsection{Liveness properties} % (fold)
\label{sub:liveness_properties}
Something desiderable will eventually happen (sooner or later)
\begin{enumerate}
	\item Starvation Freedom
\end{enumerate}
\textbf{Can be refuted by infinite behaviour}: presented typically as a loop.
% subsection liveness_properties (end)

\subsection{Fairness} % (fold)
\label{sub:fairness}
Something desirable will happen infinitely often. Whenever a subroutine takes control, it will always return it.\\
\textbf{Can be refuted by infinite behaviour}: presented typically as a loop.
% subsection fairness (end)


% section properties_of_transition_systems (end)


\section{Questions} % (fold)
\label{sec:questions}
\begin{enumerate}
	\item Is \textbf{turn} an observable variable?
\end{enumerate}
% section questions (end)
\end{document}
\grid
\grid
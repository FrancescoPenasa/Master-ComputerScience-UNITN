 \documentclass[11pt]{article}
\usepackage{listings}
\newcommand{\numpy}{{\tt numpy}}    % tt font for numpy

\topmargin -.5in
\textheight 9in
\oddsidemargin -.25in
\evensidemargin -.25in
\textwidth 7in

\begin{document}

% ========== Edit your name here
\author{Francesco Penasa}
\title{Formal methods - ch 05 symbolic CTL model Checking}
\maketitle

\medskip

% ========== Begin answering questions here
\texttt{2020 03 20}
\section{Motivations} % (fold)
\label{sec:motivations}
State space explosion: too much memory required, too much CPU time required to explore each state. Solution: symbolic representation system, we manipulate sets of states (rather than single states), expansion of sets of transitions (rather than single transitions.
% section motivations (end)

\section{Ordered Binary Decision Diagrams} % (fold)
\label{sec:ordered_binary_decision_diagrams}
OBDDs 
\begin{enumerate}
	\item if-then-else binary direcet acyclic graphs (DAGs)
	\item Variable ordering, variable are ordered in the read. 
\end{enumerate}
\paragraph{Ordered Decision Tree} % (fold)
\label{par:ordered_decision_tree}
From root to leaves, variables are encountered always in the same order.
On this tree we apply dynamic programming style of reduction and removing redundancies we get to a point where we have a OBDD.

\textit{ite == if then else}

\paragraph{} % (fold)
\label{par:}
An OBDD is a canonical representation of a boolena formula: once the variable ordering is established, equivalent formulas are represented by the same OBDD. 
\textbf{EQUIVALENCE CHECK REQUIRES CONSTANT TIME!}

% paragraph ordered_decision_tree (end)
% section ordered_binary_decision_diagrams (end)
\end{document}
\grid
\grid



\documentclass[11pt]{article}
\usepackage{listings}
\newcommand{\numpy}{{\tt numpy}}    % tt font for numpy

\topmargin -.5in
\textheight 9in
\oddsidemargin -.25in
\evensidemargin -.25in
\textwidth 7in

\begin{document}

% ========== Edit your name here
\author{Francesco Penasa}
\title{Bioinformatics Resources - biological db 2}
\maketitle

\medskip

% ==========
\texttt{2020 03 17}
\paragraph{UniProt} % (fold)
\label{par:uniprot}
Resource of protein sequences and functional informations. 
% paragraph uniprot (end)

\section{Exercises and examples} % (fold)
\label{sec:exercises_and_examples}
\paragraph{Pratical Example } % (fold)
\label{par:pratical_example_}
\begin{enumerate}
	\item Look for the ELAVL1 gene. How many human proteins there are ? How many are reviewed ?  \textbf{3 and 1}
	\item Which type of domains are in the protein that is the first result of our query ? RRM 1 RRM 2 RRM 3
	\item How many isoforms of the protein are present ? Which is their mass ? 2 and 36,092; 38,996
	\item How many protein-protein interactions does ELAVL1 make ? find protein to protein interactions, click on strings, on setting change what you want then
\end{enumerate}
% paragraph pratical_example_ (end)
% section exercises_and_examples (end)


\end{document}
\grid
\grid
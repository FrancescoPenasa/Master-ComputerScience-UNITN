\documentclass[11pt]{article}
\usepackage{listings}
\usepackage{color}

\lstset{frame=tb,
language=R,
keywordstyle=\color{blue},
alsoletter={.}
}
\newcommand{\numpy}{{\tt numpy}}    % tt font for numpy

\topmargin -.5in
\textheight 9in
\oddsidemargin -.25in
\evensidemargin -.25in
\textwidth 7in

\begin{document}

% ========== Edit your name here
\author{Francesco Penasa}
\title{Biological resources - lect 4 - lab 1}
\maketitle

\medskip

% ========== Begin answering questions here
\section{Matrix} % (fold)
rectangular table of data of the same type
\label{sec:matrix}
\begin{lstlisting}[language=R]
# matrix (vector, nrow, ncol) [row, colmn] dimnames == names of row and column
matrix(1:9, nrow = 3, ncol =3, dimnames = list(c(``X'', ``Y'', ``Z''), c(``A'', ``B'', ``C'')))

# update names
colnames(x) = c(``a'',``b'',``c'')
rownames(x) = c(``a'',``b'',``c'')

# update rows or columns
cbind(c(1,2,3),c(4,5,6))
rbind(c(1,2,3),c(4,5,6))

# change the dim of a vector of a matrix
dim(x) <- c(2,3)
\end{lstlisting}


\begin{lstlisting}[language=R]
m <- cbind(c(1,2,3), c(1,2,3), c(1,2,3))
v <- c(2,4,6)

m*2   # scalar product
m*v   # scalar product
m*m   # scalar product
v*m   # scalar product
m%*%v # dot product
m%*%m # dot product
\end{lstlisting}

\section{Lists} % (fold)
\label{sec:lists}
ordered collection of data or arbitrary types
\begin{lstlisting}[language=R]
doe <- list("john", 28, F)
doe

doe <- list(name="john", age=28, married=F)
doe$name

unlist(doe)
doe
\end{lstlisting}

\section{Data Frames} % (fold)
\label{sec:data_frames}
data frame, excel table.
Rectangular table with rows and columns BUT data within each column has the same type. but different columns may have different types.
\begin{lstlisting}[language=R]
d <- data.frame(n1 = c(3,2,1), n2=c("a","b", "c"), n3=c(T,T,F))
d
class(d)
typeof(d)

# to access use parenthesis, using index or the name
d[3,2]
d["1", "n2"] # first row and second column
d[,"n2"] # just the second column

# to keep the structure, be careful about the DROP!
d[,2, drop=FALSE]
d["1" , , drop=FALSE]

# more complex subsetting
m[1:2, c(1,3)]
m[which(m[,1] > 1), ]

# update values through subsetting
d [1,1] <- 5
d [1,] <- c(5, "S", FALSE)

# access list elem
doe[1] = "you"
doe$name = "yourmother"
\end{lstlisting}

\section{Levels} % (fold)
\label{sec:levels}
Careful about levels!	 
\begin{enumerate}
	\item \textit{character} is your basic string data structure.
	\item \textit{factor} is something important for statistics.
\end{enumerate}

\begin{lstlisting}[language=R]


\end{lstlisting}


\section{Program time} % (fold)
\label{sec:program_time}
\subsection{Branching} % (fold)
\label{sub:branching}
\begin{lstlisting}[language=R]
x <- c(1,2,3)
if (any(x>2)){
	x <- x+1
} else {
	x <- x+2
}
\end{lstlisting}
% subsection branching (end)

\subsection{Loops} % (fold)
\label{sub:loops}
\begin{lstlisting}[language=R]
for(i in 1:10){
	print(i*i)
}

i <- 1
while(i<=5){
	print(i*i)
	i = i+ sqrt(i)
}
\end{lstlisting}


\begin{lstlisting}[language=R]
# apply, sapply, lapply
x <- matrix(1:9, nrow = 3, ncol =3, dimnames = list(c("X", "Y", "Z"), c("A", "B", "C")))

# apply(arr, margin, fct)

# sum all rows
apply(x, 1, sum)

# sum all columns 
apply(x, 2, sum)

# lapply: generization for lists
lapply


# sapply: create a data structure that it is the simplest as possible
li <- list("t", "s", "q")
sapply(li, toupper)
li

fct <- function(x){ return(c(x, x*x, x*x*x))}
sapply(1:5, fct)
\end{lstlisting}
% subsection loops (end)

\section{Import data} % (fold)
\label{sec:import_data}
\begin{lstlisting}[language=R]
x <- read.delim("filename.txt")
write.table(x, file="s")

# read csv
csv <- read.csv(file)
\end{lstlisting}

\paragraph{String manipulation} % (fold)
\label{par:string_manipulation}

\begin{lstlisting}[language=R]
lines <- readLines("Table.txt")
strsplit(lines[[1]], "\t")

writeLines(x, "~/Test.txt", sep="\t")
\end{lstlisting}

\section{Storing data} % (fold)
\label{sec:storing_data}

\begin{lstlisting}[language=R]
save(x, file="x.RData")
\end{lstlisting}

% section storing_data (end)

\section{Plot} % (fold)
\label{sec:plot}
plot()
\begin{lstlisting}[language=R]
x <- seq(-pi, pi, -.1)
plot(x, sin(x))

x <- c(1, 3, 5, 7, 10, 4, 1)
barplot(x)

x <- c(1, 3, 5, 7, 10, 4, 1)
y <- c(1, 3, 5, 7, 10, 4, 1)+3
boxplot(x, y)

# scatter plot
plot(x,y)

# ggplot to see more
\end{lstlisting}

% section plot (end)


\section{Basic statistic in R} % (fold)
\label{sec:basic_statistic_in_r}
\paragraph{Qualitative data} % (fold)
\label{par:qualitative_data}
\begin{lstlisting}[language=R]
# import package
library(MASS)
# watch dataset head
head(painters)

# project the distribution of elements
table(painters$School)

pie(table(painters$School))


# complicate this, count the occurencies of the compose the data
table(painters$School, painters$Composition)
\end{lstlisting}

\paragraph{Quantitative} % (fold)
\label{par:quantitative}
\begin{lstlisting}[language=R]
library(MASS)
head(faithful, n=3)

duration = faithful$eruption
stuff...


plot(faithful$eruptions, faithful$waiting)
\end{lstlisting}

\paragraph{Statistic tools} % (fold)
\label{par:statistic_tools}
\begin{enumerate}
	\item mean
	\item median	
	\item quartile
	\item ...
	\item ...
\end{enumerate}

\begin{lstlisting}[language=R]
# perform correlation coefficient
cor.test(faithful$eruptions, faithful$waiting)
\end{lstlisting}

\paragraph{Probability distributions} % (fold)
\label{par:probability_distributions}
\begin{lstlisting}[language=R]
# normal distribution
vals <- rnorm(1200, mean=72, sd=15.2)
hist(vals)
\end{lstlisting}

% paragraph probability_distributions (end)

% paragraph statistic_tools (end)

% paragraph quantitative (end)

% section basic_statistic_in_r (end)



\paragraph{help} % (fold)
\label{par:help}
help(function)
? function
\texttt{cran} official repository
\texttt{bioonductor} biology related repository
% paragraph help (end)
\end{document}
\grid
\grid
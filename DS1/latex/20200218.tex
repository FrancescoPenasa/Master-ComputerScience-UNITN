\documentclass[11pt]{article}
\usepackage{listings}
\newcommand{\numpy}{{\tt numpy}}    % tt font for numpy

\topmargin -.5in
\textheight 9in
\oddsidemargin -.25in
\evensidemargin -.25in
\textwidth 7in

\begin{document}

% ========== Edit your name here
\author{Francesco Penasa}
\title{Distributed System 1}
\maketitle

\medskip

% ========== Begin answering questions here
\texttt{18/02/2020}
\paragraph{Synchronization in Distributed Systems} % (fold)
\label{par:synchronization_in_distributed_systems}
Time plays a fundamental role in many applications.
\begin{enumerate}
	\item Synchronize (physical) clocks to approximate global time
	\item Abstract time with logical clocks capturing (causal) ordering
	\item Collect global state
	\item Coordinate access to shared resources.
\end{enumerate}
% paragraph synchronization_in_distributed_systems (end)

\section{Synchronizing Physical Clocks} % (fold)
\label{sec:synchronizing_physical_clocks}
\paragraph{To guarantee synch} % (fold)
\label{par:to_guarantee_synch}
\begin{enumerate}
	\item Maximum \textbf{clock drift} $\rho$ is a costant
	\item maximum allowed \textbf{clock skew} $\delta$ is a parameter
	\item if two clocks drift in opposite directions, during a time inteval $\Delta T$ they accumulate a skew of $2\rho \Delta t$, resynch need $\delta / 2 \rho$ seconds.
\end{enumerate}

Synchroniza all clocks against a single one (accuracy) \textbf{OR} synchronize all clocks among themselves (agreement).
\textbf{Time monocity must be preserved} (we cannot step backward a clock).


\subsection{Protocols} % (fold)
\label{ssub:protocols}
\begin{enumerate}
	\item Server-based soution: periodically each client update himself with a server clock adding the offset $\theta$. \textbf{Problem}: time might run \textbf{backwards} on the client machine, therefore, introduce change gradually; travel time of the message.

	\item Network Time Protocol (NTP): protocol to synch time over large-scale networks \texttt{ntp.org}. The NTP servers are in Hierarchical structured (as DNS). Synchronization mechanisms: multicast (over LAN), procedure-call mode (server-based), symmetric mode(for higher levels; server exchange their roles and exchange the time).
\end{enumerate}
% subsection protocols (end)

\subsection{Observation} % (fold)
\label{sub:observation}
\begin{enumerate}
	\item Often is sufficient to agree on a time, even if it is not accurate.
	\item what matters is ordering and causality (relative order). (example: allarm -> fire; fire -> allarm).
	\item if there is no interaction, no synchronization is required.
\end{enumerate}
% subsection observation (end)


\section{Modeling a distributed execution} % (fold)
\label{sec:modeling_a_distributed_execution}
A distributed algorithm can be modeled as a collection of distributed automata.
\paragraph{relevant event in distributed algorithms} % (fold)
\label{par:relevant_event_in_distributed_algorithms}
\begin{enumerate}
	\item send event: \textit{send(m,p)}
	\item receive event: \textit{receive(m)}
	\item local event: everything else (set a variable, write a file...)
\end{enumerate}
% paragraph relevant_event_in_distributed_algorithms (end)

\paragraph{Histories} % (fold)
\label{par:histories}
\begin{enumerate}
	\item local history: history of a process $p_i$ is a sequencec of events $h_i = e^0_i e^1_i\ldots$
	\item partial history ($h^k_i$): up to event $e^{k}_{i}$
\end{enumerate}
% paragraph histories (end)

\paragraph{Happens-before} % (fold)
\label{par:happens_before}
causality, it is useful only if we don't have global time $e \rightarrow e'$. Can be measured if $e$ and $e'$ are executed by the same process; $e$ is the send and $e'$ is the corresponding receive. Of course it is transitive. If we can not say happens before then it is concurrent.
% paragraph happens_before (end)
% section modeling_a_distributed_execution (end)

\paragraph{Logical clocks} % (fold)
\label{par:logical_clocks}
enable coordination among processes without synchronization of physocal clocks, essetially a counter.
\textbf{Definition:} Logical clock $LC$ is a map function for the events $e$ of the history $H$ to an element of a time domain $T$
\[
	LC : H \rightarrow T
\]
\textbf{Clock consistency:} events could be concurrent\[
	e \rightarrow r' \Rightarrow LC(e) < LC(e')
\]
\textbf{Strong clock consistency} events can not be concurrent\[
	e \rightarrow e' \Leftrightarrow LC(e) < LC(e')
\]
% paragraph logical_clocks (end)

\paragraph{Scalar Clocks} % (fold)
\label{par:scalar_clocks}
How to assign logical clocks in a way that guarantees \textbf{clock consistency}.\\
\textbf{Definition Scalar logical clocks:} an increasing counter, each process has its own logical clock, messages carry the timestamp\\
\textbf{Update rule (+1 if is internal, max(mine,received + 1 if is receive))} which guarantees clock consistency by design.
% paragraph scalar_clocks (end)

\paragraph{Partial vs Total Order} % (fold)
\label{par:partial_vs_total_order}
Scalar clocks achieve only partial ordering, to reach total ordering we add the process ID to clocks and we use a lexicographic order.


\section{Examples} % (fold)
\label{sec:examples}
\subsection{Totally Ordered Multicast} % (fold)
\label{sub:totally_ordered_multicast}

% subsection totally_ordered_multicast (end)
% section examples (end)

% paragraph partial_vs_total_order (end)
\section{Questions} % (fold)
\label{sec:questions}

\paragraph{Scalar clock EXERCISE} % (fold)
\label{par:scalar_clock_exercise}
put the number on the events...
% paragraph scalar_clock_exercise (end)

\paragraph{what are the problems on the server-based solutions?} % (fold)
\label{par:q1}
\begin{enumerate}
	\item \textbf{Minor:} There is already a mismatch for the "travel" time, to fix this we can use timestamps T1, T2, T3, T4 to measure the RTT.
	\item \textbf{Major:} We could broke monotonicity of time, to fix the clock we should slow down until the time is fixed.
\end{enumerate}
% q1 (end)

\paragraph{Definition of Happen-before} % (fold)
\label{par:definition_of_happen_before}
We say that an event $e$ happens-before an event $e'$, and write $e \rightarrow e'$, if one of the following three cases is true:
\[
	\exists p_i \in \prod : e = e_i^r, \ e'= e^s_i,\ r < s
\]
\[
	e = send(m,*) \land e' = receive(m) 
\]
\[
	\exists e'': e \rightarrow e'' \rightarrow e' 
\]
% paragraph definition_of_happen_before (end)


% section questions (end)
% section synchronization_in_distributed_systems (end)
\end{document}
\grid
\grid
\documentclass[11pt]{article}
\usepackage{listings}
\newcommand{\numpy}{{\tt numpy}}    % tt font for numpy

\topmargin -.5in
\textheight 9in
\oddsidemargin -.25in
\evensidemargin -.25in
\textwidth 7in

\begin{document}

% ========== Edit your name here
\author{Francesco Penasa}
\title{Algorithms for Bioinformatics - lab1}
\maketitle

\medskip
Some bash cmds
\begin{lstlisting}
locate empty # find a file by name

file empty.txt # recognize the type of data contained in a file

less <filename> # visualize the content
head -n <filename> # show the first n lines

cmd1 | cmd2 | ... | cmdN # concat cmds and the out of cmdN-1 becames the input of cmdN

ls -l > list_files.txt # redirect output
ls >> list_files.txt # append the output 

grep <what> <where> # search for occurences
grep -A3 -B9 <what> <where> # search after line 3 and before line 9
grep -C3 <what> <where> # search before and after line 3
grep -i <what> <where> # case sensitive
grep -v <what> <where> # everything that does not match
grep -c <what> <where> # count lines that match
grep -w <what> <where> # match whole word
grep -n <what> <where> # display the number of matches 

whatis ls ; man ls
\end{lstlisting}

\section{Bash exercise} % (fold)
\label{sec:bash_exercise}
\begin{lstlisting}
# Lets go to your home folder 
cd /home/Francesco 

# Create a folder named “tmp”
mkdir tmp  

# Create a file that contains the manual of the ls command
man ls > manual.txt 

# Put this file into the “tmp” folder
mv manual.txt tmp/manual.txt  

# Count how many occurrences of “ls” are present (case sensitive, whole word) 
grep -i -w ls tmp/manual.txt 

# Print the first 7 lines that matched (case sensitive, not whole word) 
grep -i ls tmp/manual.txt | head -n 7 

# Save the output to a file named “10.txt” 
grep -i ls tmp/manual.txt | head -n 7 > 10.txt 

# Print the first 3 lines of the last 5 lines that matched 
grep -i ls tmp/manual.txt | tail -n 5 | head -n 3

# Concatenate this output to the file “10.txt”
grep -i ls tmp/manual.txt | tail -n 5 | head -n 3 >> 10.txt 

# Now remove the “tmp” folder
rm -r tmp 
\end{lstlisting}

% section bash_exercise (end)

\end{document}
\grid
\grid
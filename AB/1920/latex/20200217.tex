\documentclass[11pt]{article}
\usepackage{listings}
\newcommand{\numpy}{{\tt numpy}}    % tt font for numpy

\topmargin -.5in
\textheight 9in
\oddsidemargin -.25in
\evensidemargin -.25in
\textwidth 7in

\begin{document}

% ========== Edit your name here
\author{Francesco Penasa}
\title{Algorithms for Bioinformatics}
\maketitle

\medskip

% ========== Begin answering questions here

\paragraph{What is the role of algorithms in Biology/Bioinformatics?} % (fold)
\label{par:paragraph_name}

Procedure that solves a problem
\begin{enumerate}
	\item solve pratical problems.
	\item model a phenomenom $+$ simulation.
	\item consider the biological system a computational device.
\end{enumerate}
% paragraph paragraph_name (end)

\paragraph{Types of problem we will work on} % (fold)
\label{par:types_of_problem_we_will_work_on}

% paragraph types_of_problem_we_will_work_on (end)
\begin{enumerate}
	\item strings:  sequences on alphabet
	\item alignment
\end{enumerate}

\section{Turing Machines} % (fold)
\label{sec:turing_machines}
simple explanation of turing machines\\
\texttt{turing.org.uk/book/update/tmjavar.html}
% section turing_machines (end)


\section{Problems} % (fold)
\label{sec:problems}
\subsection{Alignment} % (fold)
\label{sub:alignment}
\paragraph{DNA:} % (fold)
\label{par:dna}
AGTC 
\begin{enumerate}
	\item Compare sequences for different \textit{organism}.
	\item Align them with a \textbf{query}, to find the active regions (or active parts) relevant for this sequence. (example: \textbf{BLAST})
	\item Global alignment (pairwise alignment): include the sequences completely.
	\item Local alignment (pairwise alignment): include the sequence partially.
\end{enumerate}
1st: \texttt{ACGTCCCATG}\\
2nd: \texttt{TCGCCCTG}\\
$global \rightarrow 0CG1CCC2TG$ 0==mismatch 1==insertion 2==deletion\\
$local \rightarrow CGCCCTG$ i'm happy with this and i dont care about what is not aligned.\\
The algorithm used maximize a \textbf{score}.
\subparagraph{How to define the score?} % (fold)
\label{subp:how_to_define_the_score_}
\begin{enumerate}
	\item insertion
	\item deletion
	\item mismatch 
	\item match
\end{enumerate}
with this we already moved to the \textbf{model a phenomenom} part.
The actual number differ wrt the phenomenom and wrt the time it happens (the first insertion is different than the second).
This issue of define the score lead us to the definition of \textbf{score Matrices}
% paragraph how_to_define_the_score_ (end)

% paragraph dna (end)
% subsection alignment (end)
% section problems (end)

\end{document}
\grid
\grid
\documentclass[11pt]{article}
\usepackage{listings}
\usepackage{color}

\usepackage{graphicx}
\graphicspath{ {./images/} }
% R style
\lstset{frame=tb,
language=R,
keywordstyle=\color{blue},
alsoletter={.}
}

\newcommand{\numpy}{{\tt numpy}}    % tt font for numpy

\topmargin -.5in
\textheight 9in
\oddsidemargin -.25in
\evensidemargin -.25in
\textwidth 7in

\begin{document}

% ========== Edit your name here
\author{Francesco Penasa}
\title{Algorithms for Bioinformatics - lect 5}
\maketitle

\medskip

% ========== Begin answering questions here
\texttt{2020 03 09}
\section{Summary of previous lectures} % (fold)
\label{sec:summary_of_previous_lectures}
\begin{enumerate}
	\item TMs \texttt{http://morphett.info/turing/turing.html}
	\item global sequence alignment \texttt{http://experiments.mostafa.io/public/needleman-wunsch/}
	\item local sequence alignment: the atcual alignment is the one we can see in the blue box. \texttt{http://insilico.ehu.es/align/}
\end{enumerate}
% section summary_of_previous_lectures (end)

\paragraph{Different alphabets for sequence alignmnet} % (fold)
\label{par:different_alphabets_for_sequence_alignmnet}
\begin{enumerate}
	\item DNA (4)
	\item RNA (4)
	\item Amino Acids (20)
\end{enumerate}
This alphabet are related through the genetic code. The alphabet of RNA is used for the translation the mRNA into proteins. Proteins are codified in triplets
% paragraph different_alphabets_for_sequence_alignmnet (end)

\paragraph{PAM} % (fold)
\label{par:pam}
Used to obtain scores for the substitution of amino acids, a model of mutation and acceptance based on a Markov model.\\ $m_j = $ mutability of the amino acid $j$.\\ 
$A_{jk}$ = count of accepted point mutation between amino acids $i$ and $j$.
\[
	P_{jk} = m_j \frac{A_{jk}}{\sum_{h \neq j} A_{jh}} j \neq k
\]\[
	P_jj = 1 - m_j
\]
% paragraph pam (end)


\section{Substitution Matricess: BLOSUM} % (fold)
\label{sec:substitution_matricess_blosum}
Based on "blocks" within protein sequences obtained by clustering them at a given level of similarity.
% section substitution_matricess_blosum (end)
\paragraph{Block} % (fold)
\label{par:block}
\begin{lstlisting}
	P A P A 
	A A A C
	A A C C 
	A A P A 
	A A C C 
	A A P C
s=6   w=4   s*w=24   w*s(s-1)/2=60
\end{lstlisting}
Amino acid
\begin{enumerate}
	\item A ; 14 ; 14/24
	\item P ; 4 ; 4/24
	\item C ; 6 ; 6/24
\end{enumerate}

Aligned Pair (things are considered for column)
\begin{enumerate}
	\item AA ; 26 ; 26/60
	\item AP ; 8 ; 8/60
	\item ...
	\item PP ; 3 ; 3/60 (we can see it in the thrid column)
	\item PC ; 6 ; 6/60 (we can see it in the thrid column)
	\item CC ; 7 ; 7/60
\end{enumerate}

Expected proportion: provided the frequency of the single amino acid this the proportion we expect to observe in terms of pairs. The expected proportion is estimated from the data.\\
$$2log_2 \frac{observed}{expected}$$
\begin{enumerate}
	\item PP ; 16/576 ; 1.70
	\item PC ; 48/576 ; 0.53
\end{enumerate}
Then we scale the ratio by 2 and round it to form the following substitution matrix.
\begin{lstlisting}
	A 	P 	C
A 	1 	-1 	2
P 	-1 	2 	1
C 	2 	1 	2
\end{lstlisting}

\paragraph{Cluestring of similar sequences within a block} % (fold)
\label{par:cluestring_of_similar_sequences_within_a_block}
Similar sequence segments are clustered together at a level of similarity n\%.
An additional sequence segment can be included in the cluster if it is similar at the n\% to at least ONE of the sequence segments that are already in the cluster.
% paragraph cluestring_of_similar_sequences_within_a_block (end)

\subsection{Formulas} % (fold)
\label{sub:formulas}
\begin{enumerate}
	\item asd
	\item asd
	\item asd
\end{enumerate}
% subsection formulas (end)

\section{PAM vs BLOSUM} % (fold)
\label{sec:pam_vs_blosum}
\begin{enumerate}
	\item evolutionary model: explicit \textbf{vs} none evolution model to motivate the approach.
	\item data: starting point multiple sequence alignments \textbf{vs}
	\item structure: trees \textbf{vs} clustering
	\item evolutionary distance: markov model \textbf{vs} clustering of sequences
	\item matrices: \textbf{vs}
	\item parameters: \textbf{vs}
	\item biophysical properties: \textbf{vs}
\end{enumerate}
% section pam_vs_blosum (end)

% paragraph block (end)
\end{document}
\grid
\grid
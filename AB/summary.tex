\documentclass[11pt]{article}
\usepackage{listings}
\usepackage{color}

\usepackage{graphicx}
\graphicspath{ {./images/} }
% R style
\lstset{frame=tb,
language=R,
keywordstyle=\color{blue},
alsoletter={.}
}

\newcommand{\numpy}{{\tt numpy}}    % tt font for numpy

\topmargin -.5in
\textheight 9in
\oddsidemargin -.25in
\evensidemargin -.25in
\textwidth 7in

\begin{document}

% ========== Edit your name here
\author{Francesco Penasa}
\title{Algorithms for Bioinformatics - summary}
\maketitle

\medskip

% ========== Begin answering questions here

\texttt{https://www.youtube.com/watch?v=PdyARRNwi7I}
\section{Global sequence alignment} % (fold)
\label{sec:global_sequence_alignment}
\texttt{https://en.wikipedia.org/wiki/Needleman\%E2\%80\%93Wunsch\_algorithm}
\subsection{Exercise} % (fold)
\label{sub:exercise}
\begin{enumerate}
	\item We have got two sequences $s_1$ and $s_2$ and the following weights $match = 1$ $mismatch = -1$ $gap = -2$;
	\item Put the sequences in a matrix like in table~\ref{tab:init_NW};
	\item Init the first row and the first column like in table~\ref{tab:init_rw_NW}
	\item Starting from the $0$ on the top-left (with $i=1$ and $j=1$) find the max between the following equations: 
	\begin{enumerate}
	 	\item $if\  M[i+1][label] == M[label][j+1]$ $then\ M[i][j] + match$
	 	\item $if\  M[i+1][label] \neq M[label][j+1]$ $then\ M[i][j] + mismatch$
	 	\item $M[i][j] + gap$
 	\end{enumerate} 
 	\item Repeat row to row
 	\item Since it is global the best alignment will result in the max score at the bottom right of the matrix.
\end{enumerate}
% subsection exercise (end)

\begin{table}[ht]
	\caption{Init table $s_1 = GCATGCU$ $s_2 = GATTACA$}
	\label{tab:init_NW}
	\centering

	\begin{tabular}{l|c ccccccc}
	\hline

	\hline
		  & - & G & C & A & T & G & C & U \\
	\hline
	 	- &   &   &   &   &   &   &   &  \\
	 	G &   &   &   &   &   &   &   &  \\
	 	A &   &   &   &   &   &   &   &  \\
	 	T &   &   &   &   &   &   &   &  \\
	 	T &   &   &   &   &   &   &   &  \\
	 	A &   &   &   &   &   &   &   &  \\
	 	C &   &   &   &   &   &   &   &  \\
	 	A &   &   &   &   &   &   &   &  \\
	

	\hline
	\end{tabular}
\end{table}
\begin{table}[ht]
	\caption{Init first row and first column $s_1 = GCATGCU$ $s_2 = GATTACA$}
	\label{tab:init_rw_NW}
	\centering

	\begin{tabular}{l|c ccccccc}
	\hline

	\hline
		  & - & G & C & A & T & G & C & U \\
	\hline
	 	- & 0  & -2& -4& -6& -8&-10&-12&-14\\
	 	G & -2 &   &   &   &   &   &   &   \\
	 	A & -4 &   &   &   &   &   &   &   \\
	 	T & -6 &   &   &   &   &   &   &   \\
	 	T & -8 &   &   &   &   &   &   &   \\
	 	A & -10&   &   &   &   &   &   &   \\
	 	C & -12&   &   &   &   &   &   &   \\
	 	A & -14&   &   &   &   &   &   &   \\
	

	\hline
	\end{tabular}
\end{table}

\begin{table}[ht]
	\caption{First row iteration $s_1 = GCATGCU$ $s_2 = GATTACA$}
	\label{tab:fstrow_NW}
	\centering

	\begin{tabular}{l|c ccccccc}
	\hline

	\hline
		  & - & G & C & A & T & G & C & U \\
	\hline
	 	- & 0  & -2& -4& -6& -8&-10&-12&-14\\
	 	G & -2 &$\nwarrow 1$ & $\leftarrow -1$ & $\leftarrow -3$ & $\leftarrow -5$ &  $\leftarrow \nwarrow -7$ & $\leftarrow -9$ & $\leftarrow -11$ \\

	 	A & -4 &   &   &   &   &   &   &   \\
	 	T & -6 &   &   &   &   &   &   &   \\
	 	T & -8 &   &   &   &   &   &   &   \\
	 	A & -10&   &   &   &   &   &   &   \\
	 	C & -12&   &   &   &   &   &   &   \\
	 	A & -14&   &   &   &   &   &   &   \\
	

	\hline
	\end{tabular}
\end{table}

\begin{table}[ht!]
	\caption{Second row iteration $s_1 = GCATGCU$ $s_2 = GATTACA$}
	\label{tab:sndrow_NW}
	\centering

	\begin{tabular}{l|c ccccccc}
	\hline

	\hline
		  & - & G & C & A & T & G & C & U \\
	\hline
	 	- & 0  & -2& -4& -6& -8&-10&-12&-14\\
	 	G & -2 &$\nwarrow 1$ & $\leftarrow -1$ & $\leftarrow -3$ & $\leftarrow -5$ &  $\leftarrow \nwarrow -7$ & $\leftarrow -9$ & $\leftarrow -11$ \\

	 	A & -4 & $ \nwarrow \uparrow -1$ &  $\nwarrow 0$ & $\nwarrow 0$ & $\leftarrow -2$ & $\leftarrow -4$ & $\leftarrow -6$ & $\leftarrow -8$ \\

	 	T & -6 &   &   &   &   &   &   &   \\
	 	T & -8 &   &   &   &   &   &   &   \\
	 	A & -10&   &   &   &   &   &   &   \\
	 	C & -12&   &   &   &   &   &   &   \\
	 	A & -14&   &   &   &   &   &   &   \\
	

	\hline
	\end{tabular}
\end{table}
% section global_sequence_alignment (end)































\newpage
\section{Local sequence alignment} % (fold)
\label{sec:local_sequence_alignment}
\texttt{https://en.wikipedia.org/wiki/Smith\%E2\%80\%93Waterman\_algorithm}\\
%https://en.wikipedia.org/wiki/Smith%E2%80%93Waterman_algorithm
Indian explains things: \texttt{https://www.youtube.com/watch?v=QphFHG9tmOY}\\
As Needleman Wunsch for global alignment we use a table. There are only three differences:
\begin{enumerate}
	\item The table initialization is done with all 0 as in table~\ref{tab:SW_init}
	\item We have gap penality to incentivize not starting the alignment. To implement this gap penality we will use an additional option as we can see in the following equation.
	\item We search for the max number in the table, we don't look only at the last cell.
\end{enumerate}
\newpage

\[ H = matrix \]
\[ H_{i,0} = \emptyset \ \ \ 0 \leq i \leq n \]
\[ H_{0,j} = \emptyset \ \ \ 0 \leq j \leq m  \]
\[
	H(i,j) = max \left\{\begin{array}{@{}lr@{}}
		0   	& \\
		H_{i-1,j-1} + w(a_i, b_i) & Match/Mismatch \\
		H_{i-1,j} + w(a_i, -) & Insertion \\
		H_{i,j-1} + w(-, b_i) & Deletion \\
	\end{array}\right\}
\]
\[ w = gap\_weigth\]
In the original paper
\[
	w = 1 + \frac{1}{3} - k 
\]
\begin{table}[ht]
	\caption{Smith Waterson Init first row and first column $s_1 = GCATGCU$ $s_2 = GATTACA$}
	\label{tab:SW_init}
	\centering

	\begin{tabular}{l|c ccccccc}
	\hline

	\hline
		  & - & G & C & A & T & G & C & U \\
	\hline
	 	- & 0  & 0 & 0 & 0 & 0 & 0 & 0 & 0 \\
	 	G & 0  &   &   &   &   &   &   &   \\
	 	A & 0  &   &   &   &   &   &   &   \\
	 	T & 0  &   &   &   &   &   &   &   \\
	 	T & 0  &   &   &   &   &   &   &   \\
	 	A & 0  &   &   &   &   &   &   &   \\
	 	C & 0  &   &   &   &   &   &   &   \\
	 	A & 0  &   &   &   &   &   &   &   \\
	\hline
	\end{tabular}
\end{table}

\begin{table}[ht]
	\caption{Smith Waterson first row compute $s_1 = GCATGCU$ $s_2 = GATTACA$}
	\label{tab:SW_frtrow}
	\centering

	\begin{tabular}{l|c ccccccc}
	\hline

	\hline
		  & - & G & C & A & T & G & C & U \\
	\hline
	 	- & 0  & 0 & 0 & 0 & 0 & 0 & 0 & 0 \\
	 	G & 0  & $\nwarrow 1$ & 0 & 0 & 0 & $\nwarrow 1$ & 0 & 0 \\
	 	A & 0  &   &   &   &   &   &   &   \\
	 	T & 0  &   &   &   &   &   &   &   \\
	 	T & 0  &   &   &   &   &   &   &   \\
	 	A & 0  &   &   &   &   &   &   &   \\
	 	C & 0  &   &   &   &   &   &   &   \\
	 	A & 0  &   &   &   &   &   &   &   \\
	\hline
	\end{tabular}
\end{table}

% section local_sequence_alignment (end)





























\newpage
\section{Substitution Matrices} % (fold)
\label{sec:substitution_matrices}
\texttt{https://en.wikipedia.org/wiki/Substitution\_matrix}
In bioinformatics and evolutionary biology, a substitution matrix describes the rate at which one character in a sequence changes to other character states over time. Substitution matrices are usually seen in the context of amino acid or DNA sequence alignments, where the similarity between sequences depends on their divergence time and the substitution rates as represented in the matrix



\subsection{PAM} % (fold)
\label{sub:pam}
\texttt{https://en.wikipedia.org/wiki/Point\_accepted\_mutation}
% subsection pam (end)
A point accepted mutation (PAM) is the replacement of a single amino acid in the primary structure of a protein with another single amino acid, which is accepted by the processes of natural selection.
A PAM matrix is a matrix where \textbf{each} column and row represents one of the \textbf{twenty} standard amino acids. PAM matrices are use as substitutionm matrices to score \textbf{sequence alignments for proteins}. 

\paragraph{} % (fold)
\label{par:}
Each entry in a PAM matrix indicates the \textbf{likelihood of the amino acid} of that row \textbf{being replaced with the amino acid} of that column \textbf{through} a series of one or more point accepted mutations during a specified evolutionary interval, \textbf{rather than} these two amino acids \textbf{being aligned due to chance}.
In proteins the alphabet goes to 20 (from the 4 of DNA).


\subsubsection{Construction of PAM matrices} % (fold)
\label{ssub:constructiono_of_pam_matrices}
Each PAM matrix is designed to compare two sequences which are a specific number of PAM units apart. Two sequences $S1$ and $S2$ are at evolutionary distance of PAM1, if $S1$ has converted to $S2$ with an average of one amino acid substitution per 100 amino acids.

For the $j$th amino acid, the values $m(j)$ and $f(j)$ are its mutability and frequency. The mutability of an amino acid is the ratio of the number of mutations it is involved acceptably with A as the matrix.
\[
	m(j) = \frac{\sum^{20}_{i=1, i \neq j} A(i,j)}{n(j)}
\] 
The frequency of the amino acids are normalised so that they sum to 1.
If the total number of occurrences of the $j$th amino acid is $n(j)$ and $N$ is the total number of all amino acids, then 
\[
	f(j) = \frac{n(j)}{N}
\]

The mutation matrix $M$ is composed as follow.
\[
	M(i,j) = \frac{\lambda A(i,j)}{N f(j)}
\]\[
	M(j,j) = 1 - \lambda m(j)
\] for the diagonals.
The PAM matrix tells us if something is more probable than random to mutate or not.
$PAM2 = PAM1 \times PAM1$ simple matrix multiplication



\begin{table}[ht]
	\caption{PAM matrix}
	\label{tab:PAM_example}
	\centering

	\begin{tabular}{l|ccccc}
	\hline

	\hline
		   & C & S & T & ... & W \\
	\hline
	 	C  &   &   &   &   & \\
	 	S  &   &   &   &   & \\
	 	T  &   &   &   &   & \\
	 	...&   &   &   &   & \\
	 	W  &   &   &   &   & \\
	\hline
	\end{tabular}
\end{table}






















\subsection{BLOSUM} % (fold)
\label{sub:blosum}
\texttt{https://en.wikipedia.org/wiki/BLOSUM}\\
\texttt{https://www.youtube.com/watch?v=xDUzRTx12ZE}\\
Protein scoring matrices < statistics of conservation, substitution of characters in nature

\textbf{consisten with natures.}
Identical amino acids > any substitution
conservative substitutions > non-conservative substitutions 

Higher identity == lower evolutionary distance

Provide a likelihood for a character substitution

\section{Main steps in the construction of a BLOSUM matrix} % (fold)
\label{sec:main_steps_in_the_construction_of_a_blosum_matrix}
\begin{enumerate}
	\item Eliminate repeated sequences in each block of the cluster of blocks whose percentage of identity is at most K.
	\item Pair column by column, perserving the order of the block. (each pair is counted twice), store the column in a table 
	\item Store the total count of pairs in a matrix. $C(X,Y) = $ total number of $XY$ pairs in the sample block.
	\item Calculate the frequency of each pair, each number is to be divided by the total number of possible pairs. number of all possible pairs $T =  0.5[ColumnsN \times RowsN(RowsN-1)]$ 
	\item compute the (frequency) matrix $Q$. $Q(X, Y) = C(X,Y)/ T$ 
	\item find the expected probability $P(X) = Q(X, X) + 0.5 \sum_{Y \neq   X} Q(X,Y)$
	\item calculate the expected frequencies $E(X,X) = P(X)^2$ $E(X,Y) = 2P(X)P(Y)$
	\item the log-odds ration $L(X,Y) = log_2[Q(X,Y)/E(X,Y)]$
	\item scale an round of the values.  $B(X,Y) = round[2 \times L(X,Y)]$
\end{enumerate}

\begin{table}[ht]
	\caption{step 1 and step 2}
	\label{tab:tablename}
	\centering

	\begin{tabular}{l|ccc}
	\hline

	\hline
		Seq1 & K & P & T\\
		Seq2 & K & P & V\\
		Seq3 & V & P & A\\
		Seq4 & T & P & V\\
		Seq5 & T & P & K\\
	\hline

	\hline
	\end{tabular}
\end{table}

\begin{table}[ht]
	\caption{step 3}
	\label{tab:tablename2}
	\centering

	\begin{tabular}{l|c|c}
	\hline
		\textbf{Pairs} & \textbf{Pairs in column 1} & \textbf{Total} \\
	\hline
		KK & 1 & 5\\
		KT or TK & 4 & 7\\
		KV or VK & 2 & 9\\
		TT & 1 & 6\\
		TV or VT & 2 &  8 \\
	\hline

	\hline
	\end{tabular}
\end{table}
% section main_steps_in_the_construction_of_a_blosum_matrix (end)

% subsection blosum (end)
% section substitution_matrices (end)
\end{document}
\grid
\grid
\documentclass[11pt]{article}
\usepackage{listings}
\usepackage{color}

\usepackage{graphicx}
\graphicspath{ {./images/} }
% R style
\lstset{frame=tb,
language=R,
keywordstyle=\color{blue},
alsoletter={.}
}

\newcommand{\numpy}{{\tt numpy}}    % tt font for numpy

\topmargin -.5in
\textheight 9in
\oddsidemargin -.25in
\evensidemargin -.25in
\textwidth 7in

\begin{document}

% ========== Edit your name here
\author{Francesco Penasa}
\title{Algorithms for Bioinformatics - lect 4}
\maketitle

\medskip

% ========== Begin answering questions here
\section{Global sequence alignment} % (fold)
\label{sec:global_sequence_alignment}
\texttt{https://en.wikipedia.org/wiki/Needleman\%E2\%80\%93Wunsch\_algorithm}
\subsection{Exercise} % (fold)
\label{sub:exercise}
\begin{enumerate}
	\item We have got two sequences $s_1$ and $s_2$ and the following weights $match = 1$ $mismatch = -1$ $gap = -2$;
	\item Put the sequences in a matrix like in table~\ref{tab:init_NW};
	\item Init the first row and the first column like in table~\ref{tab:init_rw_NW}
	\item Starting from the $0$ on the top-left (with $i=1$ and $j=1$) find the max between the following equations: 
	\begin{enumerate}
	 	\item $if\  M[i+1][label] == M[label][j+1]$ $then\ M[i][j] + match$
	 	\item $if\  M[i+1][label] \neq M[label][j+1]$ $then\ M[i][j] + mismatch$
	 	\item $M[i][j] + gap$
 	\end{enumerate} 
 	\item Repeat row to row
 	\item Since it is global the best alignment will result in the max score at the bottom right of the matrix.
\end{enumerate}
% subsection exercise (end)

\begin{table}[h]
	\caption{Init table $s_1 = GCATGCU$ $s_2 = GATTACA$}
	\label{tab:init_NW}
	\centering

	\begin{tabular}{l|c ccccccc}
	\hline

	\hline
		  & - & G & C & A & T & G & C & U \\
	\hline
	 	- &   &   &   &   &   &   &   &  \\
	 	G &   &   &   &   &   &   &   &  \\
	 	A &   &   &   &   &   &   &   &  \\
	 	T &   &   &   &   &   &   &   &  \\
	 	T &   &   &   &   &   &   &   &  \\
	 	A &   &   &   &   &   &   &   &  \\
	 	C &   &   &   &   &   &   &   &  \\
	 	A &   &   &   &   &   &   &   &  \\
	

	\hline
	\end{tabular}
\end{table}
\begin{table}[h]
	\caption{Init first row and first column $s_1 = GCATGCU$ $s_2 = GATTACA$}
	\label{tab:init_rw_NW}
	\centering

	\begin{tabular}{l|c ccccccc}
	\hline

	\hline
		  & - & G & C & A & T & G & C & U \\
	\hline
	 	- & 0  & -2& -4& -6& -8&-10&-12&-14\\
	 	G & -2 &   &   &   &   &   &   &   \\
	 	A & -4 &   &   &   &   &   &   &   \\
	 	T & -6 &   &   &   &   &   &   &   \\
	 	T & -8 &   &   &   &   &   &   &   \\
	 	A & -10&   &   &   &   &   &   &   \\
	 	C & -12&   &   &   &   &   &   &   \\
	 	A & -14&   &   &   &   &   &   &   \\
	

	\hline
	\end{tabular}
\end{table}

\begin{table}[h]
	\caption{First row iteration $s_1 = GCATGCU$ $s_2 = GATTACA$}
	\label{tab:fstrow_NW}
	\centering

	\begin{tabular}{l|c ccccccc}
	\hline

	\hline
		  & - & G & C & A & T & G & C & U \\
	\hline
	 	- & 0  & -2& -4& -6& -8&-10&-12&-14\\
	 	G & -2 &$\nwarrow 1$ & $\leftarrow -1$ & $\leftarrow -3$ & $\leftarrow -5$ &  $\leftarrow \nwarrow -7$ & $\leftarrow -9$ & $\leftarrow -11$ \\

	 	A & -4 &   &   &   &   &   &   &   \\
	 	T & -6 &   &   &   &   &   &   &   \\
	 	T & -8 &   &   &   &   &   &   &   \\
	 	A & -10&   &   &   &   &   &   &   \\
	 	C & -12&   &   &   &   &   &   &   \\
	 	A & -14&   &   &   &   &   &   &   \\
	

	\hline
	\end{tabular}
\end{table}

\begin{table}[h!]
	\caption{Second row iteration $s_1 = GCATGCU$ $s_2 = GATTACA$}
	\label{tab:sndrow_NW}
	\centering

	\begin{tabular}{l|c ccccccc}
	\hline

	\hline
		  & - & G & C & A & T & G & C & U \\
	\hline
	 	- & 0  & -2& -4& -6& -8&-10&-12&-14\\
	 	G & -2 &$\nwarrow 1$ & $\leftarrow -1$ & $\leftarrow -3$ & $\leftarrow -5$ &  $\leftarrow \nwarrow -7$ & $\leftarrow -9$ & $\leftarrow -11$ \\

	 	A & -4 & $ \nwarrow \uparrow -1$ &  $\nwarrow 0$ & $\nwarrow 0$ & $\leftarrow -2$ & $\leftarrow -4$ & $\leftarrow -6$ & $\leftarrow -8$ \\

	 	T & -6 &   &   &   &   &   &   &   \\
	 	T & -8 &   &   &   &   &   &   &   \\
	 	A & -10&   &   &   &   &   &   &   \\
	 	C & -12&   &   &   &   &   &   &   \\
	 	A & -14&   &   &   &   &   &   &   \\
	

	\hline
	\end{tabular}
\end{table}

% section global_sequence_alignment (end)
\newpage
\section{Local sequence alignment} % (fold)
\label{sec:local_sequence_alignment}
\texttt{https://en.wikipedia.org/wiki/Smith\%E2\%80\%93Waterman\_algorithm}\\
%https://en.wikipedia.org/wiki/Smith%E2%80%93Waterman_algorithm
\texttt{https://www.youtube.com/watch?v=QphFHG9tmOY}\\
As Needleman Wunsch for global alignment we use a table. There are only three differences:
\begin{enumerate}
	\item The table initialization is done with all 0 as in table~\ref{tab:SW_init}
	\item We have gap penality to incentivize not starting the alignment.
	\item We search for the max number in the table, we don't look only at the last cell.
\end{enumerate}

\begin{table}[h]
	\caption{Smith Waterson Init first row and first column $s_1 = GCATGCU$ $s_2 = GATTACA$}
	\label{tab:SW_init}
	\centering

	\begin{tabular}{l|c ccccccc}
	\hline

	\hline
		  & - & G & C & A & T & G & C & U \\
	\hline
	 	- & 0  & 0 & 0 & 0 & 0 & 0 & 0 & 0 \\
	 	G & 0  &   &   &   &   &   &   &   \\
	 	A & 0  &   &   &   &   &   &   &   \\
	 	T & 0  &   &   &   &   &   &   &   \\
	 	T & 0  &   &   &   &   &   &   &   \\
	 	A & 0  &   &   &   &   &   &   &   \\
	 	C & 0  &   &   &   &   &   &   &   \\
	 	A & 0  &   &   &   &   &   &   &   \\
	

	\hline
	\end{tabular}
\end{table}
% section local_sequence_alignment (end)

\newpage
\section{Substitution Matrices} % (fold)
\label{sec:substitution_matrices}
\subsection{PAM} % (fold)
\label{sub:pam}
\texttt{https://en.wikipedia.org/wiki/Point\_accepted\_mutation}
% subsection pam (end)

\subsection{BLOSUM} % (fold)
\label{sub:blosum}
\texttt{https://en.wikipedia.org/wiki/BLOSUM}
% subsection blosum (end)
% section substitution_matrices (end)
\end{document}
\grid
\grid
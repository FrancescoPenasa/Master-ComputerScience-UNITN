\documentclass[11pt]{article}
\usepackage{listings}
\usepackage{color}

\usepackage{graphicx}
\graphicspath{ {./images/} }
% R style
\lstset{frame=tb,
language=R,
keywordstyle=\color{blue},
alsoletter={.}
}

\newcommand{\numpy}{{\tt numpy}}    % tt font for numpy

\topmargin -.5in
\textheight 9in
\oddsidemargin -.25in
\evensidemargin -.25in
\textwidth 7in

\begin{document}

% ========== Edit your name here
\author{Francesco Penasa}
\title{Algorithms for Bioinformatics - lect 4}
\maketitle

\medskip

% ========== Begin answering questions here
\texttt{2020 03 04}
\paragraph{} % (fold)
\label{par:}

Till now, we have talked about global sequence alignment (NW algorithm) and local sequence Alignment (SW algorithm). In both cases we want to minimize a score, $M_{i,j}$.

\paragraph{} % (fold)
\label{par:}
Until now the alphabet we have seen is limited to the DNA (ATCG), $match = 1$  $mismatch = -1$ independently from the actualt letters. Both algorithm that we have seen so far are correct, we actually find the maximum score.
% paragraph  (end)

\begin{enumerate}
	\item solve pratical problems (so far we are here, NW and SW)
	\item modeling a biological phenomenom	
	\item understanding the computation done by the biological system.
\end{enumerate}

\section{Modeling a biological phenomenom} % (fold)
\label{sec:modeling_a_biological_phenomenom}
\texttt{http://xaktly.com/GeneticCode.html} here we can see the abbrevietions for the aminoacid, this means that so far we use only one alphabet but we could use even the aminoacid one.
% section modeling_a_biological_phenomenom (end)

\begin{lstlisting}
AUG AUU GGU
AUG AUU GGC	
\end{lstlisting}
this seems different but due to the fact that the last triplet represents the same aminoacid the actual difference is 0.
With this alphabet we don't expect that the match-mismatch model works in the same optimal way.

\subsection{Substitution matrices} % (fold)
\label{sub:substitution_matrices}
\begin{enumerate}
	\item PAM (1978) \texttt{https://en.wikipedia.org/wiki/Point\_accepted\_mutation} Percent Accepted Mutation.
	\item BLOSUM 
\end{enumerate}



\begin{lstlisting}
AAU
AAA
\end{lstlisting}
\begin{enumerate}
	\item Generation
	\item Conservation (acceptation)
\end{enumerate}


\begin{figure}[tb]
	\centering
	\includegraphics[scale=0.3]{image.jpg}
	\caption{For each aminoacid we  have the probability of be substituted by another aminoacid, basically a markov chain.}
	\label{fig:figure1}
\end{figure}
% subsection substitution_matrices (end)



Let's assume we have this kind of sequence

\begin{lstlisting}
ACGH DBGH ADIJ CBIJ
(ACGH)-ABGH-(DBGH) (ADIJ)-ABIJ-(CBIJ)
(ABGH)-AB**-(ABIJ)
\end{lstlisting}

\begin{table}[tb]
	\caption{accepted point mutation}
	\label{tab:tablename}
	\centering

	\begin{tabular}{l|cccccccc}
	\hline

	\hline
		 & A & B & C & D & G & H & I & J \\
	\hline
		A&   &   & 1 & 1 & & & & \\
		B&   &   & 1 & 1 & & & & \\
		C& 1 & 1 &  & & & & & \\
		D& 1 & 1 &  & & & & & \\
		G&  &  &  & & & & & \\
		H&  &  &  & & & & & \\
		I&  &  &  & & & & & \\
		J&  &  &  & & & & & \\
	\hline

	\hline
	\end{tabular}
\end{table}

\[
	A_{jk} = \frac{1} {nT} \sum_{T} A_{jk}^{T}
\]
\[
	P_{jk}^{(1)} = m_j \frac{A_jk}{\sum_{h \neq j} A_{jh}}
\]
\[
	P_{jj}^{(1)} = 1 -m_j 
\]
\[
	m_j = \frac{1}{n p_j z} \frac{\sum {l \neq j }A_j l}{\sum_h \sum_{l \neq h} A_{hl} }
\]

\begin{enumerate}
	\item $m_j = $ the mutability of the amminoacid $j$ 
	\item $z$ is a normalizing factor s.t.$ = \sum_{ji}(P_j \mu_j) = \frac{1}{100}$
\end{enumerate}

The steps we needed were, find probabilities and find the scores.
\[
	q_{jk}^{(n)} = P_j p_{jk}^{(n)}
\]
\[
	S_{[j,k]}^n = \lambda log \frac{q_{jk}} {P_j P_k} = \lambda log \frac{P_{jk}^{(n)}}{P_k}
\]
j and k are aminoacid, p their propabilities,
q is the probability that j mutated in a certain way, the score is the comparison.

\[
	S_{[j,k]}^{(n)} \neq S_{[k,j]} ^{(n)}
\]

the PAM100 matrix represents 100\% change, the PAM250 matrix represents 250\% change.
They were determined by the global alignment of sequences that differ by less than 85\%. One PAM represents a 1\% change in all residues or one Point Accepted Mutation per 100 residues

\end{document}
\grid
\grid
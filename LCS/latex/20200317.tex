\documentclass[11pt]{article}
\usepackage{listings}
\newcommand{\numpy}{{\tt numpy}}    % tt font for numpy

\topmargin -.5in
\textheight 9in
\oddsidemargin -.25in
\evensidemargin -.25in
\textwidth 7in

\begin{document}

% ========== Edit your name here
\author{Francesco Penasa}
\title{Laboratory of Computer Science Education}
\maketitle

\medskip

% ========== Begin answering questions here
\texttt{2020 03 17}
\paragraph{} 
\label{par:1}
Partire da un problema, partendo dai livelli piu' alti.
Teaching informatics in secondary or even primary schools has to start with programming. Learning programming means learning a language of communication with technical systems, learning to tell a machine what activity we would like to have from it.

\paragraph{} 
\label{par:2}
L’Informatica è la scienza che studia il trattamento dell’informazione nelle sue varie forme. È un potente mezzo per riuscire a risolvere qualsiasi problema posto (in qualsiasi ambito) in maniera ottimale, veloce e creativa.

Lo scienziato dell’Informazione (comunemente «Informatico») ha il ruolo quindi di definire i passi che portano alla risoluzione di un problema, per poi «addestrare» una macchina a compierli per ottenere la soluzione automaticamente.

L'informatica alla fine ci apre la mente (secondo me), penso sia molto utile per ragionare velocemente ed avere un pensiero fluido (oltre computazionale)


\end{document}
\grid
\grid
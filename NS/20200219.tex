\documentclass[11pt]{article}
\usepackage{listings}
\newcommand{\numpy}{{\tt numpy}}    % tt font for numpy

\topmargin -.5in
\textheight 9in
\oddsidemargin -.25in
\evensidemargin -.25in
\textwidth 7in

\begin{document}

% ========== Edit your name here
\author{Francesco Penasa}
\title{Network Security - Network aspects}
\maketitle

\medskip

% ========== Begin answering questions here
\texttt{https://securitylab.disi.unitn.it/doku.php?id=course_netsec_2016}\\

Autonomous Systems -> logically separated networks
Internet is made of several logically separated networks.

Communication protocols AS betwenn itself
\begin{enumerate}
	\item IGP?
	\item AOSP?
\end{enumerate}

Communication protocols between AS
\begin{enumerate}
	\item bgp?
\end{enumerate}

\paragraph{OSI model} % (fold)
\label{par:osi_model}
\begin{enumerate}
	\item 1 physical, prevent physical access
	\item 2 data link
	\item 3 network 
	\item 4 transport (tcp port)
	\item 5 session (ip address)
	\item 6 presentation (ethernet port)
	\item 7 application
\end{enumerate}
% paragraph osi_model (end)

\paragraph{OSI data link layer} % (fold)
\label{par:osi_data_link_layer}
Identified by ethernet address, bounded to physical interface identified by a MAC
48-bit identifier (HH-HH-HH-HH-HH-HH).
Inside local network all routing happens at MAC address (IP only outside local network)
`ifconfig en0'

ethernet cable is a shared bus, so it is supposed to be truted.

% paragraph osi_data_link_layer (end)

\paragraph{Network layer IP} % (fold)
\label{par:network_layer_ip}
IPv4 and IPv6
IP address represent a host, IP dinamically assigned by DHCP server

% paragraph network_layer_ip (end)
ARP = associate IP address to a MAC address.x`'

ICMP = internet control message protocol


stateful protocol (TCP)

stateless protocol (UDP)




\end{document}
\grid
\grid
Bayesian decision theory part 2

Slide: univariate nprmal case unknown unknown \mu known \sigma ^2

Slide: conjugate priors
ive got a likelihood function p(x|\sigma)

what is a prior?

when the prior and likelihood have the same form we can use the same distribution

conjugate prior for the likelihood 

if the likelihood is gautian and we are modeling, the mean the conjugate prior of the gautian is again a gautian.

// multinomial == bag of coins



Slide: bernoulli distribution

p(D|\theta) = \product p(xi|\theta)

in a bernoulli the maximum likelihood parameters is the number of the outcomes over the possible outocomes..`.

in a bernoulli with the bayesian estimatiln
	the probability of head given the dataset was the fraction of tim in which head was observed either in the dataset or virtual sample. (\alpha head)



slides: multinomial distribution
substantially a dice.

one-hot encoding
we cab nidek a vaurabke x wutg a vectir z 
where zi(x) = 1 iff x = xi

z)x3_ = {0,0,1,0,0,0}

p(x |\thera) = \product (k,r_) \theta k ^ zk(x)


conjugate prior of the multinomial is the dirichletd distribution


D = {R,R,R,G,B,G,B,R,B,G}

P(D| \theta) = \theta r \theta r \theta r \theta g \theta b \theta g \theta b \theta r..`.
 = \theta r ^ 4 \theta g ^ g \theta ^ b = \product \theta i ^ Ni


the prior of the mean is a gautian
the prior of the variance is a gamma

N comprare cuffie
\documentclass[11pt]{article}
\usepackage{listings}
\newcommand{\numpy}{{\tt numpy}}    % tt font for numpy

\topmargin -.5in
\textheight 9in
\oddsidemargin -.25in
\evensidemargin -.25in
\textwidth 7in

\begin{document}

% ========== Edit your name here
\author{Francesco Penasa}
\title{Agent-Oriented Software Engineering - 02\_Intrduction\_to\_Agents\_and\_Multi-Agent\_Systems}
\maketitle

\medskip

% ========== Begin answering questions here
\texttt{2020/02/18}\\
\texttt{02 to slide 30}\\
Computers are not very good at knowing what to do, then we elaborate a runtime reaction for unpredictable situation.
\paragraph{Capability} % (fold)
\label{par:capability}
\begin{enumerate}
	\item elaborate a new plan
	\item \textbf{tradeoff}: limited time to react
\end{enumerate}
% paragraph capability (end)

\textbf{We can't anticipate all the possible situation but we can think about a software that can react them}.
\textit{Example: software that can decide itself what to do (stock market bot, diagnostic software for medicine)}



\paragraph{Delegate the activity of the software} % (fold)
\label{par:delegate_the_activity_of_the_software}
behave and act on human behalf, interconnection + distribution + delegation
\begin{enumerate}
	\item Social abilities
\end{enumerate}
% paragraph delegate_the_activity_of_the_software (end)

\paragraph{Agent and its environment} % (fold)
\label{par:agent_and_its_environment}
An agent react to some observation of the environment that have been done.
% paragraph agent_and_its_environment (end)

\section{Agent attributes} % (fold)
\label{sec:agent_attributes}
\begin{enumerate}
	\item proactiveness
	\item reactivity
	\item social ability
\end{enumerate}
% section agent_attributes (end)

\section{Role/goal vs. Task} % (fold)
\label{sec:role_goal_vs_task}
We assign \textit{tasks} to software.
We assign \textit{roles/goals} to agents
Higher level of conceptualization
% section role_goal_vs_task (end)

\section{Multi-agent} % (fold)
\label{sec:multi_agent}
In most cases, a signle agent is not enough.
Agents interact with one-another.
To successfully interact, they will require the ability to \textbf{cooperate, coordinate and negotiate} with each other. (examples of robots that moves an object)

\begin{enumerate}
	\item Interaction: between agents
	\item local and organization interest: relationships and rules
	\item organizational structure
\end{enumerate}

\textbf{Maximize the cohesion}: intra-actions within a subsystem
\textbf{Minimize the coupling}: inter-actions among subsystems.

% section multi_agent (end)

\section{Example of agents} % (fold)
\label{sec:example_of_agents}
\begin{enumerate}
	\item Any control system can be viewed as an agent (thermostat)
\end{enumerate}
% section example_of_agents (end)

We can explicit goals without exlpicit tell how to communicate thanks to the role (example robot soccer)

We don't make difference between human and software.
Agent = computer in \textbf{some envirnment}, capable of \textbf{flexible autonomous} (reactive and proactive) actions to reach goals.
\end{document}
\grid
\grid
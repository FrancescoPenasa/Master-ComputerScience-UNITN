\documentclass[11pt]{article}
\usepackage{listings}
\newcommand{\numpy}{{\tt numpy}}    % tt font for numpy

\topmargin -.5in
\textheight 9in
\oddsidemargin -.25in
\evensidemargin -.25in
\textwidth 7in

\begin{document}

% ========== Edit your name here
\author{Francesco Penasa}
\title{AOSE - multi-agent}
\maketitle

\medskip

% ========== Begin answering questions here
\texttt{09/03/2020 da slide 40 in poi }
\section{Logical architecture} % (fold)
\label{sec:logical}
\begin{enumerate}
	\item we define the internal state we some logical formula
	\item we define the deduction rules.
\end{enumerate}
% section logical (end)

\section{Reactive architecture} % (fold)
\label{sec:reactive_architecture}
\begin{enumerate}
	\item by using some priority rules we can specify each behaviour of the agent
	\item rules specified by layered, higher priority to lower layers.
	\item example: avoid obstacles -> wander -> explore -> build maps -> monitor changes -> identify objects; 
	\item example: Steels Mars Explorer system (see last time slides).
	\item simple, no reasoning, low computational complexity and robust.
\end{enumerate}
% section reactive_architecture (end)

\section{Hybrid architectures} % (fold)
\label{sec:hybrid_architectures}
\begin{enumerate}
	\item Integrate to the reactive architecture some aspects of the logic architectures
	\item at first Reactive part: react to the environment
	\item then deliberative part: develop plans and make decision (more sophisticate reasoning)
	\item two different type of layering: horizontal layer give priority to the lover layer; vertical layer -> see layer one at the time. 
\end{enumerate}
% section hybrid_architectures (end)

\section{BDI architecture} % (fold)
\label{sec:bdi_architecture}
\begin{enumerate}
	\item The most popular one.
	\item Plans
	\item Belief: information about the world
	\item Desire(Goal): my desire state, capture why a particular piece of code is executed
	\item Intention: selected course of actions (plan instances)
\end{enumerate}
% section bdi_architecture (end)
\end{document}
\grid
\grid